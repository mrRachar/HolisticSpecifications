 
%===========================================================================
%  Definition-Lemma-Theorem-Proof


\newif\ifNumberResults\NumberResultstrue
\def\@@opargbegintheorem#1#2#3{\@@@@begintheorem{\bf\@@thmname{#1}{#2}(#3)}}
\def\@@begintheorem#1#2{\@@@@begintheorem{\bf\@@thmname{#1}{#2}}}
\def\@@@@begintheorem#1{\par\removelastskip\smallskip\noindent{#1}}
\def\@@thmname#1#2{#1\ \ifNumberResults#2\ \fi}

% similarly \Proof or \begin{Proof}...\end{Proof}
% prefer proofs with statements if possible - hence \penalty700
%\let\qedsymbol\S% make it \square or \blacksquare if you like for kb
\let\qedsymbol \Box
\def\qed{\hfill{$\qedsymbol$}}
\def\Proof{\par\removelastskip\smallskip\penalty700\noindent{\bf Proof}\enskip}
\def\endProof{\qed\penalty-700 \smallskip}
\let\endproof\endProof 
 \newtheorem{definition}{Definition} %  \newtheorem{definition}[theo]{Definition}
 \newtheorem{example}{Example} %  \newtheorem{example}[theo]{Example}
 \newtheorem{conjecture}{Conjecture} %\newtheorem{conjecture}[theo]{Conjecture}
 \newtheorem{theorem}{Theorem} %\newtheorem{theo}{Theorem}
 \newtheorem{note}{Note} % \newtheorem{note}[theo]{Note}
 \newtheorem{observation}{Observation} %  \newtheorem{observation}[theo]{Observation}


\newcommand{\kw}[1]{\bf{#1}}   
\newcommand{\lit}[1]{\prg {#1}\xspace}
 
\newcommand{\syntax}[1]{\prg{{\it #1}}}
\newcommand{\BBC}{$::=$}  
\newcommand{\SOR}{\ensuremath{\ \mid\ }}  
\newcommand{\MID}{\SPsmall ~ \mid ~ \SPsmall}  


\newcommand{\pre}{\ensuremath{_{\PRE}}}  %kjx no \sc  in math mode
\newcommand{\post}{\ensuremath{_{\POST}}}%kjx no \sc  in math mode
\newcommand{\PRE}{\pre}
\newcommand{\POST}{\post}

\newcommand{\interp}[2]{\ensuremath{\lfloor{#1}\rfloor_{#2}}}
  

\newcommand{\prg{\ensuremath{\prg{M}}}

 \newcommand{\semi}{\mbox{{\kw {;}}\ }}
 \newcommand{\lb}{\mbox{\tt{\bf{\{ }}}}
 \newcommand{\rb}{\mbox{\tt{\bf{\} }}}}
 \newcommand{\lp}{\mbox{\tt{\bf{( }}}}
 \newcommand{\rp}{\mbox{\tt{\bf{) }}}}
 
 
\newcommand{\mkpair}{\fatsemi}
\newcommand{\link}{\!\circ\!}


% --------------- identifiers ----------------------------------------------------------------------------------------------------------------

 


\newcommand{\Pol}[1] {\ensuremath{\prg{Pol}\_{\prg{#1}}}}
 
\newcommand{\strongImplies}{\leqq}  
\newcommand{\weakImplies}{\lessapprox} 
\newcommand{\frames}{~\kw{frames}~}

\newcommand{\appref}[1]{see App.~\ref{#1}}
 
\newcommand{\LangOO} {\ensuremath{{\mathcal L}ang{_{\tt oo}}}}

% ------------------------------------------------------------------
%                                             positions, separations
\newcommand{\cf}{{\it c.f.~}}

\newcommand{\Arising}[1]{{{\mathcal{A}}\textrm{\textit{rising}}(#1)}}
\newcommand{\Initial} {{{\mathcal I}\!nitial}}

\newcommand{\kwa}[1]{\mbox{\bf{#1}}}
 
\newcommand{\CanAccess}[2]{\kwa{access}(#1,#2 )} 
\newcommand{\Calls}[1] {\kwa{calls}(#1)} 
\newcommand{\Future}[1]{\kwa{will}(#1)}
\newcommand{\Using}[2]{#1\,\kwa{in}\, #2}  
\newcommand{\Past}[1]{\kwa{was}(#1)}
\newcommand{\Changes}[1]{\kwa{change}(#1)}


\newcommand{\A}{\ensuremath{A}}
\newcommand{\B}{\ensuremath{B}}
 
\newcommand{\SA}{\ensuremath{B}} 
\newcommand{\SAPrime}{\ensuremath{B'}} 

\newcommand{\SE}{\ensuremath{\prg{e}}}  
\newcommand{\SEPrime}{\ensuremath{\prg{e}'}}     
\newcommand{\SEOne}{\ensuremath{\prg{e}_1}} 
\newcommand{\SETwo}{\ensuremath{\prg{e}_2}}  
 
  
% structuring macros
\newcommand{\EndDefLemma}{\noindent $\bigtriangleup$}


 
 
%-----------------

\newcommand{\Body}[2]{\ensuremath{\mathcal{B}ody(#1,\prg{#2})}}
\newcommand{\LookUp}[2]{\ensuremath{ {#1}({\prg{#2}}) }}
\newcommand{\ClassOf}[2] {\ensuremath{{\mathcal C}{\mathit{lass}}(#1)_{#2}}}
 
\newcommand{\inset}[3]{\prg{#1}\!\in\!\prg{#2},\ldots,\prg{#3}}
 \newcommand{\AND}{\SPsmall {\mbox{and}} \SPsmall}
\newcommand{\WITH}{\SPsmall {\mbox{with}} \SPsmall}

\renewcommand{\if}{\SP {\mbox{ if }} \SP}
\newcommand{\SP}{\strut \ \ \ \ }


\newcommand{\OR}{\SPsmall {\mbox{or}} \SPsmall} 
\renewcommand{\implies}{\ensuremath{\longrightarrow}}
  

% ------------------------------------------- Infereence Rules and Tables ---------------- 

%Macros for inference rules
\newcommand{\inferencerule}[2]{
\begin{array}{l} #1 \\ \hline #2 \end{array}
}

\newcommand{\inferenceruleN}[3]
{
\begin{array}{l}
% \SP\SP\SP\SP\SP\SP\SP\SP
% \SP\SP\SP\SP\SP\SP\SP\SP
\SP\SP\SP\SP\SP\SP\SP\SP
\SP\SP\SP\SP\SP\SP  {\sf #1}
\\ #2  \\ \hline   #3
  \end{array}
}

\newcommand{\inferenceruleNN}[3]
{
\begin{array}{l}
\SP\SP\SP\SP\SP\SP\SP\SP
\SP\SP\SP\SP\SP\SP\SP\SP
\SP\SP\SP\SP\SP\SP\SP\SP
\SP\SP\SP\SP\SP\SP\SP\SP

   {\sf #1}
\\ #2  \\ \hline   #3
  \end{array}
}



%--------------------------------- the ones that Susan introduced
\newcommand{\z}{{\prg z}}

\newcommand{\Fields}[3]{\ensuremath{{\mathcal F}(}\\Mg{#1},\prg{#2},
\prg{#3}\ensuremath{)} }
\newcommand{\FieldIds}[2]{\ensuremath{{\mathcal F}{\it {s}}(\M{#1},\prg{#2})}}
\newcommand{\Meths}[3]{\ensuremath{{\mathcal M}(}\M{#1},\prg{#2},
\prg{#3}\ensuremath{)} }

 
   
%  \newcommand{\nullPEC}{\lit {nullPntrExc}}
% \newcommand{\back}{{$\!\!\!\!\!\!\!$}}


 


 

  

 
 


 

 
