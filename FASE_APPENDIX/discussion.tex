
\paragraph{Specification Language}

\kjx{this section is new, but James didn't really know what Sophia wanted.}

The key to writing and using holistic specifications is to capture the
implicit assumptions underlying the design of the system we are
specifying.  The design principle animating the design of \Chainmail\
is to make writing holistic specifications as straightforward as
possible.

This is why we support bi-directional temporal operators (``will'',
``was'' etc) rather than just one temporal direction. Some assertions
are easier to express looking forwards (if a vending machine takes a
coin it will eventually dispense a chocolate bar) while other
assertions as easier to express looking backwards (if a vending
machine dispenses a chocolate bar, it has already accepted the coin
to pay for it).

Similarly, in an imperative setting, the question of whether a change can
occur at all is at least as important as the precise details of the
change. This is why we support the ``changes'' operator in \Chainmail;
to enable specifications to capture that notion directly, rather than
e.g.\ expressing explicit differences in values between pre- and post-
states.

For modelling the open world, the interactions between the external
and internal modules are essential.   \kjx{doesn't know what more to
say here someone else can fill it in} 
We included predicates in \Chainmail to be able to state explicitly which side of the boundary between modules an object is and what can cross that boundary.

The issues we have had to deal with are those related to space in
various ways: ``access'', ``in'', and ``external''.  One object
referring to another appears simple enough on the surface, but causes
significant problems in an open world setting. In \Chainmail, these
problems appear with universal quantification on the source of a
reference: ``in'' enables us to bound these quantifications.



\paragraph{Underlying Language}

\kjx{pulled in from earlier version}

For our underlying language, we have chosen an object-oriented, class based language,
\sd{but we  expect that the ideas} will also be
applicable to other kinds of languages (object-oriented or
otherwise).
%% \sd{We  could, \eg} extend our work to prototype-based programming
%% by creating an (anonymous) class to reify each
%% prototype \cite{graceClasses}. 
%
We have chosen to use a dynamically typed language because many of the
problems we hope to address are written in these
languages: web apps and mashups in Javascript; backends in Ruby or
PHP.  We expect that supporting types would make the problem easier,
not harder, but at the cost of significantly increasing the complexity
of the trusted computing base that we assume will run our programs. In
an open world, without some level of assurance (e.g. proof-carrying
code) about the trustworthiness of type information: unfounded
assumptions about types can give rise to new vulnerabilities that
attackers can exploit \cite{pickles}.

Finally, we don't address inheritance. As a specification language,
individual \Chainmail assertions can be combined or reused without any
inheritance mechanism: the semantics are simply that all
the \Chainmail\ assertions are expected to hold at all the points of
execution that they constrain.  \LangOO\ does not contain inheritance
simply because it is not necessary to demonstrate specifications of
robustness: whether an \LangOO class is defined in one place, or
whether it is split into many multiply-inherited superclasses, traits,
default methods in interfaces or protocols, etc.\ is irrelevant,
provided we can model the resulting (flattened) behaviour of such a
composition as a single logical \LangOO\ class.

% Julian: below is some discussion about the coq proofs.
\begin{itemize}
\item
\begin{enumerate}
\item
Lemma 5 (Assertions are Classical-1)
\item
Lemma 6 (Assertions are Classical-2)
\end{enumerate}
\end{itemize}


% SD removed, as some of this is in Related work
%\paragraph{Contracts and Preconditions}
%
%Traditional specification languages based on pre- and post-conditions
%are generally based on design-by-contract assumptions: ``if the
%precondition is not satisfied, the routine is not bound to do
%anything'' \cite{meyer92dbc} --- that is, the routine can do
%anything. Ensuring preconditions is the responsibility of the caller,
%and if the caller fails to meet that responsibility it is not the
%responsibility of the invoked function to fix the problem \cite{Mey88}.
%Underlying this approach, however, is the assumption of a closed
%system: that all the modules in a system are equally trusted, and so
%inserting redundant tests would just make the system more complex and
%more buggy. Indeed, \citet{meyer92dbc} states that ``This principle is
%the exact opposite of the idea of defensive programming.''
%
%In an open world, however, we do not have the luxury of trusting the
%other components with which we interact --- indeed we barely have the
%luxury of trusting ourselves.  We must assume that our methods may be
%invoked at any time, with any combination of arguments the underlying
%platform will support, irrespective of the overall state of the
%system, or of the object that receives the method request: in some
%sense, this is the very definition of an open system in an open world.
%Since methods cannot control when they are invoked, we must work
%as if all (potentially) externally-visible methods just have the precondition
%\prg{true} --- and we had better be very careful about any assumptions
%we make about which method or objects are in fact externally visible
%and which are not.  
%%\susan{I like it too.}\kjx{likes this sentence: 
%\sd{Holistic specifications directly support robust programming by making those kinds of
%assumptions explicit, giving the necessary conditions under which
%effects may take place and objects may become accessible, and then can
%help programmers ensure their programs maintain those conditions.}
%% WAS, but SD argues it is not right
%%Holistic specifications
%%directly support robust programming by making those kinds of
%%assumptions explicit, giving the necessary conditions under which
%%objects should be accessed or their methods invoked, and then can
%%help programmers ensure their programs maintain those conditions.}

 

%%% it's a point James would like made somewhre? but where?



%% \paragraph{Necessary v.s.\ sufficient conditions.}

%% Are the necessary conditions the same as the complement of all the
%% sufficient conditions?  The \sd{possible state transitions of a component are
%% described by the transitive closure of the individual transitions.
%% % , and
%% %the reachable states are those that are reachable from initial states through
%% %this transitive closure relation.
%% Taking  Figure \ref{fig:NecessaryAndSuff} part (a), if we calculate the 
%% complement of the transitive closure 
%% of the transitions, then we could, presumably, obtain all transitions which will not happen, 
%% and if we apply this relation to the set of initial states, we obtain all 
%% states guaranteed not to be reached.
%% Thus, using the sufficient conditions, we  obtain implicitly   
%% more fine-grained information than that  available through the yellow transitions and 
%% yellow boxes in Figure \ref{fig:NecessaryAndSuff}  part (b).}

%% \sd{This implicit approach} is mathematically sound. But it is impractical, brittle with
%% regards to software maintenance, and weak with regards to reasoning in
%% the open world.


%% \sd{The implicit approach} is impractical: it suggests that when interested in
%% a necessity guarantee a programmer would need to read the
%% specifications of all the functions in that module, \sd{and think 
%% about all possible sequences of such functions and all interleavings with 
%% other modules. }
%% What if the bank did indeed enforce that only the account owner may withdraw
%% funds, but had another function which allowed the manager to appoint
%% an account supervisor, and another which allowed the account
%% supervisor to assign owners?

%% \sd{The implicit approach} is also brittle with regards to software maintenance:
%%  it gives no guidance to the team maintaining a piece of
%% software.  If the necessary conditions are only implicit in the
%% sufficient conditions, then developers' intentions about those
%% conditions are not represented anywhere: there can be no distinction
%% between a condition that is accidental (if \sd{\prg{notify}-ing} is not
%% implemented, then it cannot be permitted) and one that is essential
%% (money can only be transferred by account owners).
%% Subsequent developers may inadvertently add functions which break these
%% intentions, without even knowing they've done so.

%% Finally, the implicit approach  is   weak when it comes to reasoning
%% about programs in an open world:  it does not give any
%% guarantees about objects when they are passed as arguments to calls
%% into unknown code. For example, what guarantees can we make about the
%% top of the DOM tree when we pass a wrapper pointing to lower parts of
%% the tree to an unknown advertiser?
