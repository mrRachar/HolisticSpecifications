The DAO  {(Decentralised Autonomous Organisation)}~\cite{Dao}  is a famous Ethereum contract  which aims to support
collective management of funds,  and to place power directly in the
hands of the owners of the DAO
rather than delegate it to directors.
Unfortunately, the DAO was not robust:
a re-entrancy bug   exploited in June 2016 led  to a loss of   \$50M, and
a hard-fork in the  chain ~\cite{DaoBug}.
%
%In a similar style as that  of the ERC20 spec earlier,
%We can give a \Chainmail~specification
With holistic specifications  we can  write a succinct requirement that a
DAO contract should always be able to repay any owner's money.
Any contract which satisfies such a holistic specification cannot demonstrate the DAO bug.
 
Our specification consists of three requirements.
First, that the DAO always holds at least as 
much money as any owner's balance.
%  \james{ALL owners? or does that follow?} 
To express this we use 
the field \prg{balances} which is a mapping from participants's addresses to 
numbers. Such mapping-valued fields exist in Solidity, but they could
also be taken to be ghost fields~\cite{ghost}.
  
\vspace{.1cm}

\noindent
 \strut \hspace{0.5cm} $\forall \prg{d}:\prg{DAO}.\forall \prg{p}:\prg{Any}.\forall\prg{m}:\prg{Nat}.$\\
\strut \hspace{0.5cm} $[\ \ \prg{d.balances(p)}=\prg{m}  \ \ \  \longrightarrow  \ \ \ \prg{d}.\prg{ether}\geq \prg{m} \ \ ] $


\noindent
Second, that when an owner asks to be repaid, she is sent all her money.
\vspace{.1cm}

\noindent
 \strut \hspace{0.5cm} $\forall \prg{d}:\prg{DAO}.\forall \prg{p}:\prg{Any}.\forall\prg{m}:\prg{Nat}.$\\
\strut \hspace{0.5cm} $[\ \ \prg{d.balance(p)}=\prg{m}
 \ \wedge \ \Calls{\prg{p}}{\prg{repay}}{\prg{d}}{\_}  $\\
 $\strut \hspace{5.5cm}   \ \ \  \longrightarrow  \ \ \  \Future{\Calls{\prg{d}}{\prg{p}}{\prg{send}}{\prg{m}}}\ \ ] $  
\vspace{.1cm}

 

%\kjx{combined defn ensuring enough eth at the time repayment is made:\\
%\noindent
% \strut \hspace{0.5cm} $\forall \prg{d}:\prg{DAO}.\forall \prg{p}:\prg{Any}.\forall\prg{m}:\prg{Nat}.$\\
%\strut \hspace{0.5cm} $[\ \ \prg{d.balance(p)}=\prg{m}
% \ \wedge \ \Calls{\prg{p}}{\prg{repay}}{\prg{d}}{\_}  $\\
% $\strut \hspace{5.5cm}   \ \ \  \longrightarrow  \ \ \  \Future{\Calls{\prg{d}}{\prg{send}}{\prg{p}}{\prg{m}}
% \wedge \prg{d}.\prg{ether}\geq \prg{m}}\ \ ] $
%\vspace{.1cm}
%}
%
%\kjx{negative example for talk, upping funds not calling send:\\
%\noindent
% \strut \hspace{0.5cm} $\forall \prg{d}:\prg{DAO}.\forall \prg{p}:\prg{Any}.\forall\prg{m}:\prg{Nat}.$\\
%\strut \hspace{0.5cm} $\ \ \Calls{\prg{d}}{\prg{send}}{\prg{p}}{\prg{m}}
% \strut \ \ \  \longrightarrow  \ \ \  \prg{d}.\prg{ether}\geq \prg{m}$\\
% % SD thinks the last  \longrightarrow sgould be wedge
%  $\strut \ \ \  \longrightarrow  \ \ \  \prg{p.funds}
%  = \Prev{\prg{p.funds}} + \prg{m}$
%\vspace{.1cm}
%}


\noindent
\sd{Third, that the balance of an owner is a function of the its balance in the previous step,
or the result of it joining the DAO, or asking to be repaid \etc.}
 
\noindent
$\strut \hspace{0.5cm} \forall \prg{d}:\prg{DAO}.\forall \prg{p}.\forall:\prg{m}:\prg{Nat}.$\\
$\strut \hspace{0.5cm} [ \ \ \  \prg{d.Balance(p)}=\prg{m} \ \ \  \longrightarrow   
 \ \  \ \ 
  [ \  \ \Prev{\Calls{\prg{p}}{\prg{repay}}{\prg{d}}{\_}}\, \wedge\, \prg{m}=\prg{0} \ \ \ \ \vee $\\
$\strut \hspace{5.7cm}      
\Prev{\Calls{\prg{p}}{\prg{join}}{\prg{d}}{\prg{m}}}  \ \ \ \ \vee   $\\
 $\strut \hspace{5.7cm}  ... \  ]$ \\
%                         
%                         $ \left\{
%                            \begin{array}{ll}
%                             \prg{0}, & \hbox{if}\ Prev(Call(\prg{p},\prg{d.repay(),\_})    \\
%                             \vee
%                             \\
%                             \prg{m},  & \hbox{if}\  Prev(Call(\prg{p},\prg{d.join(),m}))   \\
%                             ..., & ...
%                           \end{array}
%                         \right.    $\\
$\strut \hspace{0.5cm} ] $
  


More cases are needed to reflect the financing and repayments of proposals, but they can be expressed with the concepts described so far.


 

\noindent
The requirement that \prg{d} holds at least \prg{m} ether precludes the DAO bug,
in the sense that  any contract satisfying that spec cannot exhibit  the  bug:   a contract
which satisfies the spec  is guaranteed to always have enough money to satisfy all \prg{repay} requests.
This guarantee  holds, regardless of how many functions there are in the DAO.
In contrast, to preclude the DAO  bug with a classical spec, one would need to write a spec for each of the
DAO functions (currently 19), a spec for each function of the auxiliary contracts used by the DAO,
and then study their emergent  behaviour.

These 19 DAO functions   have several different concerns:
who may vote   for a proposal, who is eligible to submit a proposal,
how long the consultation period is for deliberating a proposal, what
is the quorum, how to chose curators, what is the value of a token,
Of these groups of functions, only  a handful affect the balance of a
participant. Holistic specifications allow us to concentrate on aspect of DAO's behaviour across \emph{all} its functions.
 
