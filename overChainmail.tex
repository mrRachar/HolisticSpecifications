%\section{\Chainmail\ Overview}
%\label{sect:chainmail}

%KJX In this section we give a brief overview of the most salient features of our approach, and give a full exposition in the next two sections.

%\jncomment
\james{stuff shamelessly copied in from unpublished chainmail rejects}

In this section we introduce our specification language \Chainmail,
and describe how it can be used to write holistic specifications.
We define \Chainmail's semantics formally below 
sections~4~to~7432 below.




  
Unlike JML or Dafny \cite{Leavens-etal07,dafny} \Chainmail specifications are not
tied tightly to the systems they are specifying.
%\sdcomment
\sophia{Not sure what this means.}
 Rather, a system may
be described by a set of \Chainmail\ specifications, each consisting of
a number of named \lstinline+policy+ clauses.
%\sdcomment
\sophia{We do not have that 
syntax any more. Do we need it?}
\Chainmail\ specifications and policies overlap and are interlinked to
provide strong protection against attacks --- just like the links in
physical chainmail.
%\sdcomment
\sophia{What does overlap mean?}
Each policy should aim to capture one very
specific concern in the design of a system. 
%\sdcomment
\sophia{Do we want to discuss style before semantics? I can see the appeal}Traditional policies are
expressed as Hoare triples --- often describing a single method
invocation on an instance of the class being specified (as in
Figure~\ref{BANK-OR-DOM}).  Holistic policies are expressed as 
temporal or spatial invariants that a module that
conforms to the specification must maintain even though any other code
may be executed (as in policy FOO in Figure~\ref{DOM-OR-BANK}).

%% \Chainmail\ policies (and specifications) can cross cut both each
%% other and the various modules and objects in the system being
%% specified.  The validity of a specification is the conjunction of its
%% policies; a module or an object must satisfy all the specification's
%% policies for us to consider that the object meets the specification.
%% Policies and specifications are not tied to any specific module or
%% class: rather, any implementing module that satisfies the
%% specification's policies obeys the specification.


\Chainmail can express necessary specifications because its invariant
language includes \kjx{several carefully-chosen} \textit{holistic}
assertion constructs, along with traditional state-based assertions.
%\jncomment
\james{if we say things like that, need to discuss alternative
  constructs in the discussion section. do we have alternatives?
  transitive Access. two-state assertions. what else? \sd{that would be good, but 
  more work to do. Can some body do it? Is it very important to have?}}.
That is, as well as supporting assertions on the contents of local
variables and object fields (\eg $\prg{x}.\f1 > \prg{this}.\f2$),
\Chainmail\ incorporates assertions which talk about
%
\textit{access}
--- objects being accessible from other objects (\eg
$\CanAccess{\x}{\y}$);
%
\textit{control} --- the
next method to be invoked ($\Calls {\x} {\y} {\m} {\z}$);
%
\textit{authority} --- about the change of some
property (\eg $\Changes{\x.\f}$);
%
\textit{space} --- some property being observable within a subset of
the current state ( $\Using{\A}{S}$);
%
and
%
\textit{time} --- about some property
holding in the future or in the past (\eg $\Future \A$ or $\Past \A$).
%
.
%\scd
  \sophia{While
many  individual
features of \Chainmail can  be found also in other work, we
%claim that their  combination as well as
%their application in the specification of open systems are novel.
argue that their power and novelty for specifying open systems lies in their careful
  combination}
  %\kjx
  \james{Hmm. a delicate, subtle argument \ldots }
%
%
These assertions draw from some concepts from object capabilities
($\CanAccess{\_}{\_}$  for  permission and $\Changes{\_}$ for
authority) 
as well as temporal logic ($\Future \A$, $\Past \A$ and friends), and the relation of
our spatial connective ($\Using{\A}{S}$)  with ownership and effect
systems.
%\sdcomment
\sophia{TODO: add references here}. 
%\jncomment
\james{don't forget the whole
  AOP mionitoring/specs stuff, just for fun.  (HELM contracts!!! (vs
  Meyer Contracts :-)}  
%
%\jncomment
\james{somewhere, should we say something like: the goal is
  to allow wholistic specifations with as extra little machinery as
  possible over a basic Hoare language}.

\paragraph{Permission: Access}

Our first holistic assertion $\CanAccess{\x}{\y}$ asserts that one
object $\x$ has a direct reference to another object $\y$: that a
reference to $\y$ is the value of at least one of $\x$'s fields.
This assertion can be used to make assertions about heap structures (and thus
object values), for example if a \prg{cacheValid} field is true,
then that object can access to a cached value:

\begin{lstlisting}
  a.cacheValid == true `$\weakImplies \CanAccess(a)(cacheValue)$` 
\end{lstlisting}

%\kjx
\james{do we really not also need reachable (transitive closure of access)}
\susan{we don't seem to need transitivity in any of the examples}


\paragraph{Control: Calls}

The  $\Calls {\x} {\y} {\m} {\z}$
assertion is more-or-less the control flow analogue of
the access assertion, and is true 
in program states where a method on object 
${\x}$ makes a method call ${\y}.{\m}({\z})$ --- that is it calls method 
{\m} on object {\y} with arguments {\z}.


\begin{lstlisting}
   another illustrative example, presumably in the context of the
   running example from the intro. which we need to pick first.
\end{lstlisting}


\paragraph{Authority Changes}

The $\Changes{\x.\f}$  assertion is true when the value of \prg{x.f}
in the next state is different to the value in the current state.
For example, we 

\begin{lstlisting}
   another illustrative example, presumably in the context of the
   running example from the intro. which we need to pick first.
\end{lstlisting}

\paragraph{Space: With}

The space assertionn $\Using{\A}{S}$ states the some assertion $\A$ is
true when the heap used by that assertion is restrcited to the
footprint $\S$  
%\kjx
\james{footprint F?}.

%\kjx
\james{OK someone needs to explain what that's for and when/why it is sound to
do it}

\begin{lstlisting}
   another illustrative example, presumably in the context of the
   running example from the intro. which we need to pick first.
\end{lstlisting}


\paragraph{Time: Next, Will, Prev, Was}

\Chainmail supports several temporal operators familiar from temporal
logic ($\Future \A$ or $\Past \A$ or $\Next \A$ or $\Prev \$A$). 
We support birectional temporal
assertions, constraining either future ($\Future \A$. $\Next \A$)
or past behaviour ($\Past \A$.  $\Prev \$A$) either considering only
the  immediate next or immediate previous step ($\Next \A$,
$\Prev \$A$) or for conditions that become eventually true in some
distant future, or were true once in some distant past  
($\Future \A$, $\Past \A$). We have bidirectional pairs of operators
to give expressiveness in writing assertions: this does not offer any
additional reasoning power. 

For example, a part of the observer pattern is that when
a subject is notified of a change, then the observer must
be told to update itself.    We can write this from the
subject's perspective, looking forwards:

\begin{lstlisting}
  Call(_,subject,notify,_) --> Will(Call(subject,observer,update,_))
\end{lstlisting}

\noindent meaning that once notify is called on a subejct, then its
observer will be updated sometime in the future.  We can write a very
similar specification for an observer, looking backwards.

\begin{lstlisting}
  Call(subject,observer,update,_) --> Was(Call(_,subject,notify,_))
\end{lstlisting}

\noindent meaning that if a subject updates an observer, that subject
have been notified sometime previously. We could tighten each
specifaction, so that the update must immediately follow the
notification, by replacing $\Future \A$ or $\Past \A$ with 
$\Next \A$ or $\Prev \A$.

%\kjx
\james{bother. should probably do time earlier, because most of the
   example assertions I can think of a things like
   \lstinline+(change(x.f) ->  past(call(_,xm,_)+ which needs the temporal operator}


%\jncomment
\james{OK really do need the example to be chosen to do more work here for this}
