\begin{figure}[thb]
\begin{lstlisting}
class Bank{

   method newAccount(amt){
        if (amt>=0) then{
            return new Account(this,amt)
   }   }
}

class Account{

    field balance
    field myBank
    
    method deposit(src,amt){
       if (amt>=0 && src.myBank=this.myBank && src.balance>=amt) then{
           this.balance = this.balance+amt
           src.balance = src.balance-amt
   }   }
   method makeAccount(amt){
       if (amt>=0 && this.balance>=amt) then{
           this.balance = this.balance - amt;
           return new Account(this.myBank,amt)
   }    }
}
\end{lstlisting}
 \vspace*{-7mm}
\caption{$M_{BA1}$: Implementation of \prg{Bank} and \prg{Account} -- version 1}
\label{fig:BanAccImplV1}
\end{figure}

In this section we revisit the \prg{Bank}/\prg{Account} example from
 section  \ref{sect:motivate:Bank}, and show two different
 implementations, derived from Noble et al.\ \cite{arnd18} . Both implementations  satisfy the three functional specifications and the holistic assertions
 (1), (2) and (3)  shown in section \ref{sect:motivate:Bank}.
 The first version gives rise to runtime configurations as $\sigma_1$, 
 shown on the left side of Fig. \ref{fig:BankAccountDiagrams}, while the
 second version gives rise to runtime configurations as $\sigma_2$,
 shown on the right side of Fig. \ref{fig:BankAccountDiagrams}. 

 In this code, we use more syntax than the minimal syntax defined for \LangOO in Def. \ref{defONE}, as we use conditionals, and we allow nesting of expressions, e.g.\ a field read to be the receiver of a method call. Such extension can easily be encoded in the base syntax.

$\M_{BA1}$, the fist version is shown Fig. \ref{fig:BanAccImplV1}. It keeps all the information in the \prg{Account} object: namely,
the \prg{Account} contains the pointer to the bank, and the balance, while the \prg{Bank} is a pure capability, which contains
no state but is necessary for the creation of new \prg{Account}s.
%\sophia{James, you have a nice, high level description of that.\kjx{somewhere}}
In this version we have no ghost fields.

\begin{figure}[htb]
\begin{lstlisting}
class Bank{
   field ledger // a Node
   
    method deposit(dest,src,amt){
       destNd = this.ledger.find(dest)
       srcNd = this.ledger.find(src)
       srcBalance = srcNd.getBalance()
       if ( destNd =/=null && srcNd=/=null && srcBalance>=amt && amt >=0 ) then
           destNd.addToBalance(amt)
           srcNc.addToBalance(-amt)           
    }  }     
    method newAccount(amt){
      if (amt>=0)  then{
           newAcc = new Account(this);
           this.ledger = new Node(amt,this.ledger,newAcc)
           return newAcc 
    } }
   
   ghost  balance(acc){ this.ledger.balance(acc)  } 
}
\end{lstlisting}
 \vspace*{-7mm}
\caption{$M_{BA2}$: Implementation of \prg{Bank}   -- version 2}
\label{fig:BanAccImplV2a}
\end{figure}

 
\begin{figure}[htb]
\begin{lstlisting}
class Node{
   field balance
   field next   
   field myAccount
   
   method addToBalance(amt){
       this.balance = this.balance + amt
   }   
   method find(acc){
      if this.myAccount == acc then{
          return this
     } else { 
          if this.next==null then{
              return null
          } else {
              return this.next.find(acc)
    }  } } 
    method getBalance(){ return balance }
    
    ghost balance(acc){
        if (this.myAccount == acc)  then  this.balance
             else  ( if this.next==null then -1 else this.next.find(acc) )
    }
}          
\end{lstlisting}
 \vspace*{-7mm}
\caption{$M_{BA2}$: Implementation of \prg{Node}   -- version 2}
\label{fig:BanAccImplV2a}
\end{figure}

\begin{figure}[htb]
\begin{lstlisting}
class Account{

    field myBank
    
    method deposit{src,amt){
             this.myBank.deposit(this,src,amt)
    }   }    
    method makeAccount(amt){
      if (amt>=0 && this.balance>=amt) then{
           newAcc = this.myBank.makeNewAccount(0)
           newAcc.deposit(this,amt)
           return newAcc
    }    }     

   ghost balance(){ this.myBank.balance(this) }   
}
\end{lstlisting}
 \vspace*{-7mm}
\caption{$M_{BA2}$: Implementation of  \prg{Account} -- version 2}
\label{fig:BanAccImplV2b}
\end{figure}
 


% Notice that even though anybody can create a new Account, by calling the constructor from the class \prg{Account} 
$\M_{BA1}$, the second version is shown Fig. \ref{fig:BanAccImplV2a} and \ref{fig:BanAccImplV2b}. It keeps all the information 
in the \prg{ledger}: each \prg{Node} points to an \prg{Account} 
and contains the balance for this particular \prg{Account}. Here \prg{balance} is a
ghost field of \prg{Account}; the body of that declaration calls the ghost field function \prg{balanceOf} of the \prg{Bank} which in its
turn calls the ghost field function \prg{balanceOf} of the \prg{Node}. Note that the latter is recursively defined.

Note also that \prg{Node} exposes the function \prg{addToBalance(...)}; a call to this function   modifies the \prg{balance} of an \prg{Account} without requiring that the caller has access to the \prg{Account}. This might look as if it contradicted assertions (1) and  (2)
  from section \ref{sect:motivate:Bank}. However, upon closer inspection, we see that the assertion is satisfied. Remember that we employ a two-module semantics, where any change in the balance of an account is observed from one external state, to another external state. By definition, a configuration is external if its receiver is external.  However, no external object will ever have access to a \prg{Node}, and therefore no external object will ever be able to call the method \prg{addToBalance(...)}. In fact, we can add another assertion, (4), which promises that any internal object which is externally accessible is either a \prg{Bank} or an \prg{Account}.

(4)\ \  $\triangleq$\ \ $\forall \prg{o},\forall \prg{o}'.[\ \ \External{ \prg{o}}\  \wedge\ \neg (\External{ \prg{o}'}) \ \wedge\  \CanAccess{\prg{o}}{\prg{o}'}     
    \longrightarrow \ \    \hfill$ \\
  $\strut \hspace{8.3cm} 
% TODO explain:
% we no longer need Past here, as we are ion visible states 
  [\    \prg{o}:\prg{Account}\ \vee \ \prg{o}':\prg{Bank}\  \ ]\ \ \  \ \ \ \ ] \hfill $


