Software guards our secrets, our money, our intellectual property,
our reputation \cite{covern}.  We entrust personal and
corporate information to software which works in an \emph{open} world, 
where  it interacts with 
third party software of unknown provenance, possibly buggy and potentially malicious.

This means we need our software to be \emph{robust}:
to behave correctly even if  used 
by erroneous or malicious third parties.
We expect that our bank will only make payments 
from our account if instructed by us, or by somebody we have authorised, 
that space on a web given to an advertiser will not be used
to obtain access to our bank details \cite{cwe}, or that a
concert hall will not book the same seat more than once.


%% The importance of robustness has led to the design of many programming
%% language mechanisms to help developers write robust programs:
%% constant fields, private methods, ownership \cite{ownalias}
%% as well as the object capability paradigm \cite{MillerPhD},
%% and its adoption in  web systems
%% \cite{CapJavaHayesAPLAS17,CapNetSocc17Eide,DOCaT14}, and programming languages such as Newspeak
%% \cite{newspeak17}, Dart \cite{dart15},
%% Grace \cite{grace,graceClasses}, and Wyvern \cite{wyverncapabilities}.

%% While such programming language mechanisms
While language mechanisms such as constants, invariants, 
object capabilities \cite{MillerPhD}, and 
ownership \cite{ownalias} 
make it \textit{possible} to write robust
programs, they cannot \textit{ensure} that programs are robust.
Ensuring robustness is difficult because it means 
different things for different systems: perhaps
that critical operations should only be invoked with the requisite authority;
perhaps that sensitive personal information should not be leaked; 
or perhaps that resource belonging to one user should not be consumed by another.g
%
To ensure robustness, we need ways to specify what robustness means for a 
particular program, and ways to demonstrate that the particular program 
adheres to its specific robustness requirements.

In this paper we propose \Chainmail, a specification language to
express holistic specifications.
The design of \Chainmail was guided by the study of a sequence of
examples from the object-capability literature and the smart contracts world: the
membrane \cite{membranesJavascript}, the DOM \cite{dd,ddd}, the Mint/Purse \cite{MillerPhD}, the Escrow \cite{proxiesECOOP2013}, the DAO \cite{Dao,DaoBug} and
ERC20 \cite{ERC20}.  As we worked through the
examples, we found a small set of language constructs that let us
write holistic specifications across a range of different contexts.
%
In particular, \Chainmail extends 
traditional program specification languages \cite{Leavens-etal07,Meyer92} with features which talk about:
%
\begin{description}
\item[Permission: ] 
%\ \ \textbullet \ \emph{Permission}, \ie
Which objects may have access to which other objects; 
this is central since access to an object usually also grants access to the functions it provides.
%
\item[Control: ]
%\ \ \textbullet  \ \emph{Control}, \ie
Which objects called functions on other objects; this
 is useful in identifying the causes of certain effects - eg 
funds can only be reduced if the owner called a payment function.
%
%$\bullet$ \ \\emph{Authority}, \ie which objects' state or properties may change; this is useful in describing effects, such as reduction of funds.
%
\item[Time: ]
%\ \ \textbullet \ \emph{Time}:\ \ie
What holds some time in  the past, the future, and what changes with time,
\item[Space: ]
%\ \ \textbullet \ \emph{Space}:\ \ie
Which parts of the heap are considered when establishing some property, or when 
performing program execution; a concept
related to, but different from, memory footprints and separation logics,
\item[Viewpoint: ]
%\ \ \textbullet \ \emph{Viewpoints}:\ \ie
%a distinction between the objects internal to our component, and those external to it;
Which objects and which configurations are internal to our component, and which  are
external to it;
a concept related to the open world setting.
\end{description}

%% Holistic assertions usually employ several of these features. They often have the form  of a guarantee
%% that only if some property holds now will a certain effect occur in the future, or that
%% certain effects can only be caused if another property held earlier.
%% For example, if within a certain heap (space) some change is possible in the future (time), then this particular heap 
%% (space again) contains 
%% at least one object which has access (permission) to a specific other, privileged object.
%% %\james{moved around --- not sure we need this para}
%% %\susan{I think we don't so there is a paragraph I have commented out.}
%% \forget{Often, holistic assertions typically have the form of a guarantee
%% that if some property ever holds in the future, then some other property holds now.
%% For example, if within a certain heap some change is possible in the future, then this particular heap contains 
%% at least one object which has access to a specific other, privileged object.}
%% A module satisfies a holistic assertion if  
%%  the assertion is satisfied  in all runtime configurations reachable through execution of the combined code of our module and any other module.
%%   This reflects the open-world view.

While many individual features of \Chainmail can be found in other work, 
their power and novelty for specifying open systems lies in their careful combination.
The contributions of this paper are:
\begin{itemize}
\item the design of the holistic specification language \Chainmail,
\item the semantics of \Chainmail, and a Coq mechanisation of its core,
\item the application of \Chainmail  to a sequence of examples.
%\item a further validation of \Chainmail through informal proofs of adherence of code to some of these specifications.
\end{itemize}  
  
  
The rest of the paper is organised as follows:
first Section
%~\ref{sect:motivate:Bank} 
%motivates our work via an example, and then section
~\ref{sect:chainmail} presents the \Chainmail\ specification
language.  Section~\ref{sect:formal} introduces the formal model
underlying \Chainmail, and then section~\ref{sect:assertions} defines
the 
semantics of \Chainmail's assertions.
% SD the below is NOT ture
%full details are relegated toappendices.   
Section~\ref{sect:example} shows how key points of 
exemplar problems can be specified in \Chainmail,
section~\ref{sect:discussion}
discusses our design, \ref{sect:related} considers related
work, and section~\ref{sect:conclusion} concludes.
We relegate various details to appendices.
