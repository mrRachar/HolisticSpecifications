We now define the syntax and semantics of expressions and holistic assertions.
\sd{The novel, holistic, features of \Chainmail (permission, control, time, space, and viewpoint),
as well as our wish to support some form of recursion while keeping the logic of assertions classical,  introduced 
challenges, which we discuss in this section.}

 \subsection{Syntax of Assertions}
 


\begin{definition}[Assertions]  \sd{Assertions consist of (pure) expressions \e, classical assertions about the contents of heap/stack, the usual logical  connectives, as well as our holistic concepts.}
\label{def:assertions}


 $\begin{array}{lcl}
  ~  \\
 \SE  &\BBC&    \kwN{true}   ~\SOR~  \kwN{false}   ~\SOR~  \kwN{null}  ~\SOR~  \x  \   ~\SOR~  
     \   \SE=\SE    ~\SOR~ \kwN{if}\, \SE\,   \kwN{then}\,  \SE\,    \kwN{else}\, \SE    ~\SOR~  \SE.\f\lp\ \SE^* \ \rp\\
     \\
 \A &\ \BBC   &   \SE \   \mid \  \SE=\SE  \mid \   \SE:\prg{ClassId}  \ \mid \
    \SE\in\prg{S}   \mid  \  \\
    &
  &  \A \rightarrow \A  \ \mid\  \     \A \wedge \A  \ \mid\  \  \A \vee \A  \ \mid\  \ \neg A   \ \mid \ \\
  & &  \forall \x.\A  \ \mid \  \forall \prg{S}:SET.\A  \ \mid  \  \exists \x.\A  \ \mid \  \exists \prg{S}:SET.\A  \  \ \mid\   \\
 &    &  \CanAccess x y %\ \mid\  \ \Changes e 
           \ \mid\  \Calls {\prg{x}}  {\prg{m}} {\prg{x}}  {\prg{x}^*}\\
 &    &  \Next \A  \ \mid \   \Future \A \ \mid \  \Prev \A   \ \mid \ \Past \A \ \mid \\  
 &    &   \Using \SF  \A  \ \mid \  \External \x     \\
 \\
 \x, \f, \m &\BBC&  \prg{Identifier}  ~ \\
\end{array}$
\end{definition}
%% \footnote{
%% The operators $\wedge$, $\vee$,  $\neg$ and $\forall$  could have been
%% defined  through the usual shorthands, \eg, $\neg \A$ is short for
%% $\A \rightarrow \ff$ \etc, but here we give full definitions
%% instead\kjx{can we just cut this please?}}
 
 \sd{Expressions support calls with parameters  ($\e.\f(\e^*)$); these are calls to ghostfield
functions. This  supports recursion at the level of expressions; therefore, the value of  an expression  may be
undefined (either because of infinite recursion, or because the expression accessed undefined fields or variables). 
Assertions of the form   $\e$=$\e'$ are satisfied only if both $\e$ and $\e'$ are defined. Because we do not support 
recursion at the level of assertions, assertions from a classical logic (\eg $\A \vee \neg\A$ is a tautology). }
 
\sd{We will discuss evaluation of expressions in section \ref{sect:expressions}, standard assertions about heap/stack and logical
 connectives in \ref{sect:standard}, the treatment of  permission, control, space, and viewpoint  in  \ref{sect:pcsv}
the treatment of time in  \ref{sect:time}, and properties of assertions in \ref{sect:classical}.}
 \sd{The judgement $\M\mkpair \M', \sigma  \models \A$ expresses that $\A$ holds in  $\M\mkpair \M'$ and $\sigma$, and 
while $\M\mkpair \M', \sigma  \not\models \A$  expresses that $\A$ does not hold  in  $\M\mkpair \M'$ and $\sigma$.} 
 

\subsection{Values of Expressions}
\label{sect:expressions}

The value  of  an expression  is described through judgment $ \M,\, \sigma, \SE \ \hookrightarrow\  v$,
defined in  Figure \ref{fig:ValueSimpleExpressions}.
We use the configuration, $\sigma$, to read the contents of the top stack frame
% value of variables defined in the stack frame
(rule ${\sf {Var\_Val}}$) or the contents of the heap (rule
${\sf {Field\_Heap\_Val}}$). We use the module, \M, to find the  ghost field declaration corresponding to the
ghost field being used. 



The treatment of fields and ghost fields is described in rules ${\sf {Field\_Heap\_Val}}$,\\  ${\sf {Field\_Ghost\_Val}}$ and 
${\sf {Field\_Ghost\_Val2}}$.  If the field \f~ exists in the heap, then its value is returned (${\sf {Field\_Heap\_Val}}$). 
Ghost field reads, on the other hand, have the form $\e_0.\f(\e_1,...\e_n)$, and their value is
described in rule ${\sf {Field\_Ghost\_Val}}$:
%
The lookup function $\mathcal G$  (defined in the obvious way in the Appendix, Def.\ref{def:lookup})
returns the expression constituting the body for that ghost field, as defined in the class of $\e_0$.
We return  that expression
evaluated in a configuration where the formal parameters have been substituted by the values of the actual
parameters.


Ghost fields support recursive definitions. For example, imagine a module $\M_0$ with
a class \prg{Node} which has a field called \prg{next}, and which 
had a ghost field \prg{last}, which finds  the last \prg{Node} in a sequence
and is defined recursively as \\
$~ \strut \hspace{.1cm}$ \ \ \ \prg{if}\ \ \prg{this.next}=\prg{null}\  \prg{then} \ \prg{this} \ \prg{else} \ \prg{this.next.last},\\
and another ghost field \prg{acyclic}, which expresses that a sequence is acyclic,
defined recursively as \\
$~ \strut \hspace{.1cm}$ \ \ \ \prg{if}\ \ \prg{this.next}=\prg{null}\  \prg{then} \ \prg{true} \ \prg{else} \ \prg{this.next.acyclic}.\\



The relation $ \hookrightarrow$ is partial. 
For example, assume   a configuration
$\sigma_0$ where
\prg{acyc} points to a \prg{Node} whose field \prg{next} has value \prg{null}, and   
\prg{cyc} points to a \prg{Node} whose field \prg{next} has the same value as \prg{cyc}. Then,   
$\M_0,\sigma_0,\,\prg{acyc.acyclic}  \ \hookrightarrow\  \prg{true}$, but we would have no value for 
$\M_0,\sigma_0,\, \prg{cyc.last}  \ \hookrightarrow\  ...$, nor for
$\M_0,\sigma_0,\, \prg{cyc.acyclic}  \ \hookrightarrow\  ...$.

Notice also that for an expression of the form  
\prg{\e.\f}, both ${\sf {Field\_Heap\_Val}}$ and ${\sf {Field\_Ghost\_Val2}}$ could be applicable: rule ${\sf {Field\_Heap\_Val}}$
will be applied if \prg{f} is a field of the object at \prg{e}, while rule ${\sf {Field\_Ghost\_Val}}$
will be applied if \prg{f} is a ghost field of the object at \prg{e}. We expect the set of fields and ghost fields in a 
given class to be disjoint.
This allows a specification to be agnostic over whether a field is a physical field or just ghost information.
For example, assertions (1) and (2) from section  \ref{sect:motivate:Bank} talk about the \prg{balance} of an \prg{Account}. 
In module $\M_{BA1}$ (Appendix~\ref{Bank:appendix}), where we keep the balances in the account objects, this is a physical field. 
In $\M_{BA2}$ (also in Appendix~\ref{Bank:appendix}), where we keep the
balances in a ledger, this is ghost information.  
 


\begin{figure*}
{$\begin{array}{l}
\begin{array}{llll}
\inferenceruleN {True\_Val} {
}
{
 \M,\, \sigma, \kwN{true} \ \hookrightarrow\  \kwN{true}
}
& 
\inferenceruleN {False\_Val} {
}
{
 \M,\, \sigma, \kwN{false} \ \hookrightarrow\  \kwN{false}
}
&
\inferenceruleN  {Null\_Val} {
}
{
 \M,\, \sigma, \kwN{null} \ \hookrightarrow\  \kwN{null}
}
&
\inferenceruleN {Var\_Val} {
}
{
 \M,\,  \sigma, \x \ \hookrightarrow\   \sigma({\x})
}
\end{array}
\\ \\
\begin{array}{lll}
\begin{array}{l}
\inferenceruleNM{Field\_Heap\_Val} {
 \M,\,  \sigma, \SE \ \hookrightarrow\   \alpha \hspace{1.5cm} 
 \sigma(\alpha,\f)=v
}
{
 \M,\, \sigma, \SE.\f  \ \hookrightarrow\   v
}
\\
\\
\inferenceruleNM{Field\_Ghost\_Val2} {
 \M,\, \sigma, \SE.\f \lp \rp \ \hookrightarrow\   v
}
{
 \M,\, \sigma, \SE.\f   \ \hookrightarrow\   v
}
\end{array}
& &
\inferenceruleNM{Field\_Ghost\_Val}
{
~ \\
 \M,\, \sigma, \SE_0   \ \hookrightarrow\  \alpha
\\
 \M,\, \sigma, \SE_i  \ \hookrightarrow\   v_i\ \ \ \ i\in\{1..n\}
 \\
{\mathcal{G}}
(\M, {\ClassOf {\alpha} {\sigma}}, {\f}) \  =  
\ \f\lp \p_1, \ldots \p_n \rp \lb\ \SE \ \rb
  \\
  \M,\,\sigma[\p_1\mapsto v_1, .... \p_n\mapsto v_n], \SE    \hookrightarrow_{\SAF}\   v 
 }
{
 \M,\,  \sigma, \ \SE_0.\f \lp \SE_1,....\SE_n\rp \hookrightarrow   \ v
}
\\ \\
\inferenceruleNM{If\_True\_Val} 
{
 \M,\,  \sigma, \SE \ \hookrightarrow\   \prg{true}  \\
   \M,\,  \sigma, \SE_1 \ \hookrightarrow\   v  
}
{
 \M,\, \sigma, \kwN{if}\ \SE\  \kwN{then} \ \SE_1 \ \kwN{else} \ \SE_2\  \hookrightarrow  \ v
}
& &
\inferenceruleNM {If\_False\_Val} 
{
 \M,\,  \sigma, \SE \ \hookrightarrow\   \prg{false}  \\
   \M,\,  \sigma, \SE_2 \ \hookrightarrow\   v  }
{
 \M,\, \sigma, \kwN{if}\ \SE\  \kwN{then} \ \SE_1 \ \kwN{else} \ \SE_2\  \hookrightarrow\  v
}
\\ \\ 
\inferenceruleNM {Equals\_True\_Val} 
{
 \M,\,  \sigma, \SE_1 \ \hookrightarrow\    v \\
   \M,\,  \sigma, \SE_2 \ \hookrightarrow\     v 
}
{
 \M,\, \sigma, \SE_1 =  \SE_2 \hookrightarrow \prg{true}
}
& &
\inferenceruleNM {Equals\_False\_Val} 
{
 \M,\,  \sigma, \SE_1 \ \hookrightarrow\    v \\
   \M,\,  \sigma, \SE_2 \ \hookrightarrow\     v' \hspace{2cm}  v\neq v'
}
{
 \M,\, \sigma, \SE_1 =  \SE_2 \hookrightarrow \ \prg{false}
}
\end{array}
\end{array}
$}
\caption{Value of  Expressions}
\label{fig:ValueSimpleExpressions}
\end{figure*}

\subsection{Satisfaction of Assertions - standard}
\label{sect:standard}
\sd{
We now define the semantics of assertions involving expressions, the heap/stack, and logical connectives.
The semantics are unsurprising, except, perhaps the relation between validity of assertions and the values of
expressions.
}


 \begin{definition}[Interpretations for simple expressions]

For a runtime configuration, $\sigma$,    variables $\x$ or \SF, we define its interpretation as follows:

\begin{itemize}
  \item
  $\interp {\x}{\sigma}$ $ \triangleq$ $\phi(\x)$  \ \ if \ \ $\sigma$=$(\phi\cdot\_,\_)$
  \item
  $\interp {\SF}{\sigma}$ $ \triangleq$ $\phi(\SF)$  \ \ if \ \ $\sigma$=$(\phi\cdot\_,\_)$
  \item
    $\interp {\x.\f}{\sigma}$ $ \triangleq$ $\chi(\interp {\x}{\sigma},\f)$  \ \ if \ \ $\sigma$=$(\_,\chi)$
   \end{itemize}
\end{definition}   

 
\begin{definition}[ Basic Assertions] For modules $\M$, $\M'$,  configuration $\sigma$,  we define$:$
%validity of basic assertions: 
\label{def:valid:assertion:basic}
\begin{itemize}
\item
$\M\mkpair \M', \sigma \models\SE$ \IFF   $ \M,\,  \sigma, \SE \ \hookrightarrow\   \prg{true}$ 
\item
$\M\mkpair \M', \sigma \models\SE=\SEPrime$ \IFF there exists a value $v$ such that  $\M,\,  \sigma, \SE \ \hookrightarrow\   v$  and $ \M,\,  \sigma, \SEPrime \ \hookrightarrow\   v$.
           \item
$\M\mkpair \M', \sigma \models\SE:\prg{ClassId}$ \IFF there exists an address $\alpha$ such that \\
$\strut ~ $ \hspace{2in} \hfill   
 $ \M,\,  \sigma, \SE \ \hookrightarrow\   \alpha$, and $\ClassOf{\alpha}{\sigma}$ = \prg{ClassId}.
\item
$\M\mkpair \M', \sigma \models \SE\in \prg{S}$ \IFF there exists a value $v$ such that 
 $ \M,\,  \sigma, \SE \ \hookrightarrow\   v$, and $v \in \interp{\prg{S}}{\sigma}$.
\end{itemize}
\end{definition}

Satisfaction of assertions which contain expressions is predicated on termination of these expressions.
Continuing our earlier example,  
$\M_0\mkpair \M', \sigma_0 \models \prg{acyc.acyclic}$ holds for any $\M'$, while $\M_0\mkpair \M', \sigma_0 \models \prg{cyc.acyclic}$
does not hold, and $\M_0\mkpair \M', \sigma_0 \models \prg{cyc.acyclic}=\prg{false}$ does not hold either.
In general, when $\M\mkpair \M', \sigma  \models \prg{e}$ holds,  then $\M\mkpair \M', \sigma  \models \prg{e}=\prg{true}$ holds too.
But when $\M\mkpair \M', \sigma  \models \prg{e}$ does not hold, this does \emph{not} imply that $\M\mkpair \M', \sigma  \models \prg{e}=\prg{false}$ holds.
Finally, an assertion of the form $\e_0=\e_0$ does not always hold; for example,   $\M_0\mkpair \M', \sigma_0 \models \prg{cyc.last}=\prg{cyc.last}$ does not hold.

% \subsubsection{Logical connectives, quantifiers, space and control} 
We now define satisfaction of assertions which involve logical connectives and existential or universal quantifiers, in the standard way:

\begin{definition}[Assertions with logical connectives and quantifiers]  
%We now consider 
\label{def:valid:assertion:logical}
For modules $\M$, $\M'$, assertions $\A$, $\A'$, variables \prg{x}, \prg{y}, \prg{S},  and configuration $\sigma$, we define$:$
\begin{itemize}
\item
$\M\mkpair \M', \sigma \models \forall \prg{S}:\prg{SET}.\A$ \IFF  $\M\mkpair \M', \sigma[\prg{Q}\mapsto R] \models  \A[\prg{S}/\prg{Q}]$ \\
$\strut ~ $ \hfill for all sets of addresses $R\subseteq dom(\sigma)$, and  all \prg{Q} free in $\sigma$ and $\A$.
\item
$\M\mkpair \M', \sigma \models \exists \prg{S}:\prg{SET}\!.\,\A$ \IFF  $\M\mkpair \M', \sigma[\prg{Q}\mapsto R] \models  \A[\prg{S}/\prg{Q}]$ \\
 $\strut ~ $ \hfill  for some set of addresses $R\subseteq dom(\sigma)$, and   \prg{Q} free in $\sigma$ and $\A$.
\item
$\M\mkpair \M', \sigma \models \forall \prg{x}.\A$ \IFF
$\sigma[\prg{z}\mapsto \alpha] \models  \A[\prg{x}/\prg{z}]$ \ for all  $\alpha\in dom(\sigma)$, and  some \prg{z} free in $\sigma$ and $\A$.
\item
$\M\mkpair \M', \sigma \models \exists \prg{x}.\A$ \IFF
$\M\mkpair \M', \sigma[\prg{z}\mapsto \alpha] \models  \A[\prg{x}/\prg{z}]$\\
$\strut ~ $ \hfill for some  $\alpha\in dom(\sigma)$, and   \prg{z} free in $\sigma$ and $\A$.
\item
$\M\mkpair \M', \sigma \models \A \rightarrow \A' $ \IFF  $\M\mkpair \M', \sigma \models \A $ implies $\M\mkpair \M', \sigma \models \A' $
\item
$\M\mkpair \M', \sigma \models  \A \wedge \A'$   \IFF  $\M\mkpair \M', \sigma \models  \A $
and $\M\mkpair \M', \sigma \models  \A'$.
\item
$\M\mkpair \M', \sigma \models  \A \vee \A'$   \IFF  $\M\mkpair \M', \sigma \models  \A $
or $\M\mkpair \M', \sigma \models  \A'$.
\item
$\M\mkpair \M', \sigma \models  \neg\A$   \IFF  $\M\mkpair \M', \sigma \models  \A $
does not hold.
\end{itemize}
\end{definition}

Satisfaction is not preserved with growing configurations; for example, the assertion $\forall \x. [\ \x : \prg{Account} \rightarrow \x.\prg{balance}>100\ ]$ 
may hold in a smaller configuration, but not hold in an extended configuration. 
Nor is it preserved with configurations getting smaller; consider \eg $\exists \x. [\ \x : \prg{Account} \wedge \x.\prg{balance}>100\ ]$.

\noindent
Again, with our earlier example,  
$\M_0\mkpair \M', \sigma_0 \models \neg (\prg{cyc.acyclic}=\prg{true})$    and  
$\M_0\mkpair \M', \sigma_0 \models  \neg (\prg{cyc.acyclic}=\prg{false})$, 
and also 
$\M_0\mkpair \M', \sigma_0 \models  \neg (\prg{cyc.last}=\prg{cyc.last})$
hold.

\label{sect:pl} 
We define equivalence of  assertions in the usual sense: two assertions are equivalent if they are satisfied  in
the context of the same configurations.
Similarly, an assertion entails another assertion, iff all configurations 
which satisfy the former also satisfy the latter.  

\begin{definition}[Equivalence and entailments of assertions]
$ ~ $

\begin{itemize}
\item
$\A \subseteqq \A'\  \IFF\    \forall \sigma.\, \forall \M, \M'. \ [\ \ \M\mkpair \M', \sigma \models \A\ \mbox{ implies }\ \M\mkpair \M', \sigma \models \A'\ \ ].$
\item
$\A \equiv \A'\  \IFF\     \A \subseteqq \A' \mbox{ and }  \A' \subseteqq \A.$
\end{itemize}
\end{definition}



\begin{lemma}[Assertions are classical-1]
\label{lemma:classic}
For all runtime configurations $\sigma$,    assertions $\A$ and $\A'$, and modules $\M$  and $\M'$, we have
\begin{enumerate}
\item
$\M\mkpair \M', \sigma \models \A$\ or\ $\M\mkpair \M', \sigma \models \neg\A$
\item
$\M\mkpair \M', \sigma  \models \A \wedge \A'$ \SP if and only if \SP $\M\mkpair \M', \sigma \models \A$ and $\M\mkpair \M', \sigma  \models \A'$
\item
$\M\mkpair \M', \sigma  \models \A \vee \A'$ \SP if and only if \SP $\M\mkpair \M', \sigma  \models \A$ or  $\sigma \models \A'$
\item
$\M\mkpair \M', \sigma  \models \A \wedge \neg\A$ never holds.
\item
$\M\mkpair \M', \sigma  \models \A$ and  $\M\mkpair \M', \sigma  \models \A \rightarrow \A'$  implies
$\M\mkpair \M', \sigma  \models \A '$.
\end{enumerate}
\end{lemma}
\begin{proof} \sd{The proof of part (1) requires to first prove that for all \emph{basic assertions} \A, \\
\strut \hspace{1.1cm} (*) \ \ \ either $\M\mkpair \M', \sigma  \models \A$
or $\M\mkpair \M', \sigma  \not\models \A$.\\
We prove this using Definition \ref{def:valid:assertion:basic}.
 Then, we prove (*) for all
possible assertions, by induction of the structure of \A, and the Definitions 
 \ref{def:valid:assertion:logical}, \ref{def:valid:assertion:access}, \ref{def:valid:assertion:control}, \ref{def:valid:assertion:view},  
 \ref{def:valid:assertion:space}, and \ref{def:valid:assertion:time}.
Using the definition of $\M\mkpair \M', \sigma \models \neg\A$ from Definition  \ref{def:valid:assertion:logical} we conclude the proof of (1).

For parts  (2)-(5) the proof goes by application of the corresponding definitions from \ref{def:valid:assertion:logical}.
 }
  \end{proof}.
 
 \begin{lemma}[Assertions are classical-2]
For     assertions $\A$, $\A'$, and $\A''$ the following equivalences hold
\label{lemma:basic_assertions_classical}
\begin{enumerate}
\item
$ \A \wedge\neg \A \ \equiv \  \prg{false}$
\item
$ \A \vee \neg\A   \ \equiv \  \prg{true}$
\item
$ \A \wedge \A'  \ \equiv \  \A' \wedge \A$
\item
$ \A \vee \A'  \ \equiv \  \A' \vee \A$
\item
$(\A \vee \A') \vee \A'' \ \equiv \  \A \vee (\A' \vee\A'')$
\item
$(\A \vee \A') \wedge \A'' \ \equiv \  (\A \wedge \A')\, \vee\, (\A \wedge \A'')$
\item
$(\A \wedge \A') \vee \A'' \ \equiv \  (\A \vee \A')\, \wedge\, (\A \vee \A'')$
\item
$\neg (\A \wedge \A') \  \ \equiv \  \neg  \A   \vee\, \neg \A''$
\item
$\neg (\A \vee \A') \  \ \equiv \  \neg  \A   \wedge\, \neg \A'$
\item
$\neg (\exists \prg{x}.\A )  \  \ \equiv \  \forall \prg{x}.(\neg  \A)$
\item
$\neg (\exists \prg{S}:\prg{SET}.\A )  \  \ \equiv \  \forall \prg{S}:\prg{SET}.(\neg  \A)$
%\item
% $\neg (\exists k:\mathbb{N}.\A )  \  \ \equiv \  \forall  k:\mathbb{N}.(\neg  \A)$
%\item
%$\neg (\exists \prg{fs}:FLD^k.\A )  \  \ \equiv \  \forall \prg{fs}:FLD^k.(\neg  \A)$
\item
$\neg (\forall \prg{x}. \A)  \  \ \equiv \  \  \exists \prg{x}.\neg(\A )$
\item
$\neg (\forall \prg{S}:\prg{SET}. \A)  \  \ \equiv \  \  \exists \prg{S}:\prg{SET}.\neg(\A )$
%\item
%$\neg (\forall k:\mathbb{N}. \A)  \  \ \equiv \  \  \exists k:\mathbb{N}.\neg(\A )$
%\item
%$\neg (\forall \prg{fs}:FLD^k. \A)  \  \ \equiv \  \  \exists \prg{fs}:FLD^k.\neg(\A )$
\end{enumerate}
\end{lemma}
\begin{proof}
All points follow by application of the corresponding definitions from \ref{def:valid:assertion:logical}. % and lemma 
 \end{proof}

 

% \begin{definition}
%We say that $\sigma \vdash \A$ if for any  \x\, is free in $\A$ and any
%  any term $\x.\f_1...\f_n$ appearing in $\A$,
% the interpretation $\interp{\x.\f_1...\f_n} \sigma$ is defined.
%\end{definition}
%
%Note that if we take $n=0$ in the definition above we obtain as corollary that   all variables that appear free in $\A$ they  are in the domain of the top frame in $\sigma$.
%
%\begin{lemma}[Preservation of satisfaction] $ $
%\label{lemma:preserve:valid}
%\begin{itemize}
%\item
%If  $\sigma \vdash \A$ and $\M\mkpair \M',  \sigma \vdash \A$ and   $\sigma' \subconf \sigma$, \  then  \ $\M\mkpair \M',  \sigma' \models \A$.
%\end{itemize}
%\end{lemma}
