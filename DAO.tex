The DAO~\cite{Dao}  is a famous Ethereum contract  which aims to support
collective management of funds,  and to place power directly in the
hands of the owners of the DAO
rather than delegate it to directors.
Unfortnately, the DAO was not robust:
a re-entrancy bug   exploited in June 2016 led  to a loss of   \$50M, and
a hard-fork in the  chain ~\cite{DaoBug}.
%
%In a similar style as that  of the ERC20 spec earlier,
%We can give a \Chainmail~specification
With holistic specifications  we can  write a succinct requirement that a
DAO contract should always be able to repay any owner's money.
Any contract which satisfies such an holistic specification cannot demonstrate the DAO bug.
 
Our specification consists of three requirements.
First, that the DAO always holds at least as 
much money as any owner's balance. \james{ALL owners? or does that follow?} To express this we use 
the field \prg{balances} which is a mapping from participants's addresses to 
numbers. Such mapping-valued fields exist in Solidity, but they could
also be taken to be ghost fields\footnote{cite ghost fields}:\james{see
comments about ghost fields in ERC20}

\vspace{.1cm}

\noindent
 \strut \hspace{0.5cm} $\forall \prg{d}:\prg{DAO}.\forall \prg{p}:\prg{Any}.\forall\prg{m}:\prg{Nat}.$\\
\strut \hspace{0.5cm} $[\ \ \prg{d.balances(p)}=\prg{m}  \ \ \  \longrightarrow  \ \ \ \prg{d}.\prg{ether}\geq \prg{m} \ \ ] $

\noindent
Second, that when an owner asks to be repaid, she is sent all her money.\footnote{This
does not necessarily work well with our visible states!}
\vspace{.1cm}

\noindent
 \strut \hspace{0.5cm} $\forall \prg{d}:\prg{DAO}.\forall \prg{p}:\prg{Any}.\forall\prg{m}:\prg{Nat}.$\\
\strut \hspace{0.5cm} $[\ \ \prg{d.balance(p)}=\prg{m}
 \ \wedge \ \Calls{\prg{p}}{\prg{d}}{\prg{repay}}{\_}  $\\
 $\strut \hspace{5.5cm}   \ \ \  \longrightarrow  \ \ \  \Future{\Calls{\prg{d}}{\prg{p}}{\prg{send}}{\prg{m}}}\ \ ] $ \footnote{We also need to explain that function carry money implicitly}

\vspace{.1cm}

\noindent
Third, that the balance of an owner  the previous call was
a repayment; it is \prg{m} if  the previous call was \prg{p}
joining \prg{d} and paying in \prg{m}.
\kjx{This is scrambled - I don't understand the previous sentence/}
\se{If a previous call was \prg{p} joining \prg{d} and paying in \prg{m} then the balance of \prg{p} is \prg{m}.}

\noindent
$\strut \hspace{0.5cm} \forall \prg{d}:\prg{DAO}.\forall \prg{p}.\forall:\prg{m}:\prg{Nat}.$\\
$\strut \hspace{0.5cm} [ \ \ \  \prg{d.Balance(p)}=\prg{m} \ \ \  \longrightarrow   
 \ \  \ \ 
  [ \  \ \Prev{\Calls{\prg{p}}{\prg{d}}{\prg{repay}}{\_}}\, \wedge\, \prg{m}=\prg{0} \ \ \ \ \vee $\\
$\strut \hspace{5.7cm}      
\Prev{\Calls{\prg{p}}{\prg{d}}{\prg{join}}{\prg{m}}}  \ \ \ \ \vee   $\\
 $\strut \hspace{5.7cm}  ??? \  ]$ \\
%                         
%                         $ \left\{
%                            \begin{array}{ll}
%                             \prg{0}, & \hbox{if}\ Prev(Call(\prg{p},\prg{d.repay(),\_})    \\
%                             \vee
%                             \\
%                             \prg{m},  & \hbox{if}\  Prev(Call(\prg{p},\prg{d.join(),m}))   \\
%                             ..., & ...
%                           \end{array}
%                         \right.    $\\
$\strut \hspace{0.5cm} ] $
  


More cases are needed to reflect the financing and repayments of proposals, but they can be expressed with the concepts described so far.\footnote{We do not represent al of the
aspects of the real DAO, but the real DAO is huge.
How do we still argue the value of our approach? This kind of
contradicts our argument. \kjx{NO IT DOESN'T. IT MAKES IT. we don't
even need to talk about other things that may accept the balabnce or
anything. ALL We ned to say is if you;'re an owners, you can always be
repaid. that's it. THe rest is silence, which is precisely the point.}}


 

\noindent
%The requirement that \prg{d} holds at least \prg{m} ether precludes the DAO bug,
%in the sense that  any contract satisfying that spec cannot exhibit  the  bug:   a contract
%which satisfies the spec  is guaranteed to always have enough money to satisfy all \prg{repay} requests.
%This guarantee  holds, regardless of how many functions there are in the DAO.
%In contrast, to preclude the DAO  bug with a classical spec, one would need to write a spec for each of the
%DAO functions (currently 19) and then study their emergent  behaviour.

The DAO holds 19\footnote{SD I think more -- need to check} functions which have several different concerns:
who may vote   for a proposal, who is eligible to submit a proposal,
how long the consultation period is for deliberating a proposal, what
is the quorum, how to chose curators, what is the value of a token,
Of these groups of functions, only xxx affect the balance of a
participant\footnote{We need to study the white paper to make a good
argument.} \kjx{OK yes we would, but we're not going for this now, are we?}
 
