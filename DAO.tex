The DAO ~\cite{DAO}  is a famous Ethereum contract  aiming  to support
collective management of funds,  and to place power directly in the hands of owners the DAO 
rather than delegate to directors. 
Nevertheless, a re-entrancy bug   exploited in June 2016, lead  to a loss of   \$50M, and
a hard-fork in the  chain ~\cite{DaoBug}. 
 
%In a similar style as that  of the ERC20 spec earlier, 
%We can give a \Chainmail~specification 
Holistic specifications allow us to write specifications which avoid the bug, and guarantee that under all circumstances,
all owners will be able to be reapid their money.
Namely, we
requiring that  % holds as much ether as the sum of
% the   its clients' balances, and
 the owners'  balances may only be affected by clients joining or leaving, and
projects being approved or repaying.
We also require,
as below, that % Consider a  simplified version of the DAO~\cite{DAO}:
% It keeps the moneys of a set of clients, and will refund them when they call the function \prg{repay}. 
%
% The \RoSpec~policy   from below says:  
if  \prg{p}  has a balance of \prg{m} at a \prg{DAO} contract \prg{d},
and if \prg{p} calls \prg{repay} on \prg{d}, then  
 \prg{d} is required to hold at least \prg{m} ether at the time of that call ($\prg{d}.\prg{ether}\geq \prg{m}$), and will eventually send \prg{m} back to \prg{p} (expressed as $\Future{\Calls{\prg{d.send(p)},m}}$:

%Formally: 
\vspace{.07cm}

\noindent  
% \prg{Pol\_DAO\_withdraw} \ $\equiv$ \\ 
\strut \hspace{0.5cm} $\forall \prg{d}:\prg{DAO}.\forall \prg{p}:\prg{Any}.\forall\prg{m}:\prg{Nat}.$\\
\strut \hspace{0.5cm} $[\ \  \Calls(\prg{p},\prg{d.repay(),\_})\, \wedge\, \prg{d.Balance(p)}=\prg{m} $\\ 
\strut \hspace{0.5cm} \ \ \ $\longrightarrow$\\
\strut \hspace{0.5cm} \ \ \ $\prg{d}.\prg{ether}\geq \prg{m}\ \wedge$ $\ \Future(\Calls(\prg{d.send(p)},\prg{m}))\ \ ] $ 

\noindent
The requirement that \prg{d} holds at least \prg{m} ether precludes the DAO bug,
in the sense that  any contract satisfying that spec cannot exhibit  the  bug:   a contract
which satisfies the spec  is guaranteed to always have enough money to satisfy all \prg{repay} requests.
This guarantee  holds, regardless of how many functions there are in the DAO.
In contrast, to preclude the DAO  bug with a classical spec, one would need to write a spec for each of the 
DAO functions (currently 19) and then study their emergent  behaviour. 


\vspace{.005cm}

We can now define  what it means for \prg{p} to have a  \prg{Balance} at  \prg{d}. The \prg{Balance}  is \prg{0} if the previous call was
a repayment; it is \prg{m} if  the previous call was \prg{p} joining \prg{d} and paying in \prg{m}. More cases are needed to reflect the financing and repayments of proposals, but they can be expressed with the concepts described so far.

\noindent
$\strut \hspace{0.5cm} \forall \prg{d}:\prg{DAO}.\forall \prg{p}.\forall:\prg{m}:\prg{Nat}.$\\
$\strut \hspace{0.5cm} [ \ \ \  $\\
$\strut \hspace{0.5cm} \ \  \ \  \prg{d.Balance(p)}=\prg{m}$\ $\longrightarrow$ \ $ \left\{
                            \begin{array}{ll}
                             \prg{0}, & \hbox{if}\ Prev(Call(\prg{p},\prg{d.repay(),\_})    \\
                             \vee
                             \\
                             \prg{m},  & \hbox{if}\  Prev(Call(\prg{p},\prg{d.join(),m}))   \\
                             ..., & ... 
                           \end{array} 
                         \right.    $\\
$\strut \hspace{0.5cm} ] $                         