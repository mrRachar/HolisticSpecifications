We compare the holistic and the traditional specification of ERC20.

As we said earlier,  the holistic specification gives to account holders an
 "authorisation-guarantee": their balance cannot decrease unless they
 themselves, or somebody they had authorised, instigates a transfer of
 tokens. Moreover, authorisation is {\em not} transitive: only the
 account holder can authorise some other party to transfer funds from
 their account: authorisation to spend from an account does not confer
 the ability to authorise yet more others to spend also.
 
 With traditional  specifications, such as is shown in appendix~\ref{sect:ERC20:appendix}, to obtain the "authorisation-guarantee", 
one would need to inspect the pre- and post-conditions of {\em all} the functions
in the contract, and determine which of the functions decrease balances, and which of the functions 
 affect authorisations.
In Figure \ref{fig:classicalERC20} we outline a traditional specification for the \prg{ERC20}.
We give two specifications for \prg{transfer}, another two for \prg{tranferFrom}, and one for all 
the remaining functions. The  first specification says, \eg, that if  
 \prg{p} has sufficient tokens, and it calls \prg{transfer}, then the transfer will take place.  
The second specification says that  if \prg{p} has insufficient tokens, then 
the transfer will not take place (we assume that in this
specification language, any entities not mentioned in the pre- or post-condition 
are not affected).
 
 Similarly, we would have to give another two specifications to define the behaviour of 
if \prg{p''} is authorised and executes \prg{transferFrom}, then   the balance decreases. 
But they are {\em implicit} about the overall behaviour and the   {\em necessary} conditions,
e.g., what are all the possible actions that can cause a decrease of balance?

