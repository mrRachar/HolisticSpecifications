An earlier publication of this work, published at FASE 2020 \cite{FASE}, used an example of a Safe to
demonstrate how holistic specifications could be used to protect against dangerous implementations
of a safe. We now revisit this example, and discuss a bug that has since been discovered, along with
the nuance that this example points to in Chainmail. 

 \begin{figure}[htb]
 \begin{tabular}{lll} % {lll}
\begin{minipage}{0.45\textwidth}
\begin{lstlisting}
class Safe{
   field treasure 
   field secret 
   method take(scr){
      if (secret==scr) then {
         t=treasure
         treasure = null
         return t }  }
 }
\end{lstlisting}
\end{minipage}
  &\ \ \  \ \ \ \ \  \ \ \ \ \ \ &
\begin{minipage}{0.45\textwidth}
\begin{lstlisting}
class Safe{
   field treasure   
   field secret  
   method take(scr){
       $\mathit{... as\, version\,1 ...}$ 
   }
   method set(scr){
         secret=scr }
 }
\end{lstlisting}
\end{minipage} 
 \end{tabular}
  \vspace*{-0.95cm}
  \caption{Two Versions of the class \prg{Safe}}
 \label{fig:ExampleSafe}
 \vspace*{-0.65cm}
 \end{figure}

Consider the two implementations of a \prg{Safe} class in Figure \ref{fig:ExampleSafe}. Both implementations 
present a class with two fields:
\begin{itemize}
\item
\prg{treasure} : the object stored in the safe,  and 
\item
\prg{secret} : the password required to open the \prg{Safe} and remove the treasure.
\end{itemize}
Both implementations also include a method for using the 
 \prg{secret} to remove the \prg{treasure}: \prg{take}.
The second implementation also includes another method, \prg{set}, that allows any object to change the \prg{secret}.
This clearly violates any reasonable specification for a \prg{Safe}, and seems an ideal target for a Holistic specification.
 A classical Hoare triple describing the behaviour of \prg{take} would be:
 
  \vspace{.1in}
  
% \begin{figure}[htbp] 
(ClassicSpec)$  \ \ $  $\triangleq$
\vspace{-.1in}
\begin{lstlisting}
   method take(scr)
   PRE:   this:Safe  
   POST:  scr=this.secret$\pre$  $\longrightarrow$ this.treasure=null 
               $\wedge$
          scr$\neq$this.secret$\pre$ $\longrightarrow$  $\forall$s:Safe.$\,$s.treasure=s.treasure$\pre$
 \end{lstlisting}
%^\end{figure} 
\vspace{-.2in}

(ClassicSpec)  expresses  that knowledge of the \prg{secret} is  \emph{sufficient} %condition 
to remove the treasure, and further that  knowledge of the \prg{secret} is \emph{necessary} for
 \prg{take} to remove the \prg{treasure}. (ClassicSpec) does not however ensure that there
 is not some other means of removing the \prg{treasure}, or in the case of the second implementation
 of \prg{Safe}, changing the \prg{secret}. In order to capture such a specification, we would need 
 a holistic specification.
 
 
 
\vspace{.1in}
(HolisticSpec)\ \  $\triangleq$\\ 
$\strut \hspace{.3in}   \forall \prg{s}. % $\\ 
[\ \ \prg{s}:\prg{Safe} \wedge \prg{s.treasure}\neq\prg{null}   \wedge   \Future{\prg{s.treasure}=\prg{null}} $ \\ 
 $ \strut \hspace{4.3cm}     \longrightarrow \ \  \exists \prg{o}. [\  \External{\prg{o}} \wedge  \CanAccess{\prg{o}}{\prg{s.secret}}\ ]  \  \ ] \hfill $
\vspace{.1in}

(HolisticSpec) requires that for any \prg{Safe} with a non-null \prg{treasure}, if
that treasure is ever removed, it follows that there is some current external object 
with knowledge of the \prg{secret}. i.e. it is not possible to either forge, steal, 
or illegally modify the \prg{secret}. 

Both classes in Fig. \ref{fig:ExampleSafe} satisfy (ClassicSpec), and in the paper presented at FASE 2020 \cite{FASE}, 
we claimed that the first version satisfies (HolisticSpec) while the second does not. In fact neither implementation satisfies 
(HolisticSpec). The nuance is found in the notion of \prg{external} and \prg{internal}. The counter example to (HolisticSpec)
is where the \prg{secret} of one \prg{Safe}, say \prg{s1}, is stored as the \prg{treasure} of another, say \prg{s2}: i.e. \prg{s2.treasure = s1.secret}. 
In such a case, there might be no external objects with explicit knowledge of \prg{s1.secret} (since \prg{s2} is a \prg{Safe}, and thus considered \prg{internal}), but some object might 
know \prg{s2.secret}, and thus have a way to obtain \prg{s1.secret}, and finally removing \prg{s1.treasure}. 

The discovery of this bug identifies the nuances of \prg{external} and \prg{internal}. In the current formulation of 
Chainmail, these concepts are defined at a module level, but in the example of the Safe, are needed to be defined 
at a data-structure level. i.e. in our example above, we need to be able to say that \prg{s2} is external to \prg{s1}.
We have since reformulated the Safe example by making use of ghost fields to define what is internal and external 
to a \prg{Safe}.

 \begin{figure}[htb]
\begin{lstlisting}
class Safe{
   ghost is_internal(x) = x == this || x == secret
   field treasure 
   field secret 
   method take(scr){
      if (secret==scr) then {
         t=treasure
         treasure = null
         return t }  }
 }
\end{lstlisting}
  \vspace*{-0.95cm}
  \caption{A fixed version of the class \prg{Safe}}
 \label{fig:ExampleSafeFix}
 \vspace*{-0.65cm}
 \end{figure}