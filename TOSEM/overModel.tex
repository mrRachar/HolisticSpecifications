%\section{Overview of the \Chainmail\  formal model}
% \subsection{The Open World}

Having outlined the ingredients of our holistic specification
language, the next question to ask is: When does a module $\M$ satisfy
a holistic assertion $\A$?  More formally: when does
$\M \models \A$ 
hold? 
  
Our answer has to reflect the fact that we are dealing with an  
\emph{open  world},  where  $\M$, our module, may be
linked with \textit{arbitrary untrusted code}.
%
%
%% \sd{Note that we use the term \emph{module} to talk about repositories of code; in this work modules are mappings from
%% class identifiers to class definitions.}
%
%
To % skipped the discussion of what a module is
 model the open world, we consider
 pairs of modules, 
$\M \mkpair {\M'}$,  where $\M$ is the module 
whose code is supposed to satisfy the assertion,
and $\M'$  is  another % wused to say \textit{any}  -- but why?
 module which exercises
the functionality of $\M$. We call our module $\M$ the {\em internal} module, and
 $\M'$ the {\em external} module, which represents potential
 attackers or adversaries.
     
We can now answer the question: $\M \models \A$ 
 holds if for all further, {\em potentially adversarial}, modules $\M'$ and in  all runtime configurations $\sigma$ which may be observed as arising from the  execution of the code of $\M$ combined with that of $\M'$, the assertion $\A$ is satisfied. More formally, we define:\\
$~ \strut  \hspace{1.3in} \M \models \A \ \ \  \ \ \ \ \ \mbox{
if               } \ \ \  \ \ \  \  \forall \M'.\forall \sigma\in\Arising
{\M \mkpair  {\M'}}. [\ \M \mkpair  {\M'},\sigma \models \A\ ]$.  \\
\\
\jm{\\$~ \strut  \hspace{1.3in} \M \models \A \ \ \  \ \ \ \ \ \mbox{
if               } \ \ \  \ \ \  \  \forall \M'.\forall \sigma_0,  \sigma_0 \in \Initial{\M \mkpair \M'}, \\ ~ \strut  \hfill
\sigma\in\Arising{\M \mkpair  {\M'}, \sigma_0}. [\ \M \mkpair  {\M'}, \sigma_0 \ldots \sigma \models \A\ ]$. } \\
Module $\M'$ represents all possible clients of {\M}.  As it is arbitrarily chosen, it reflects the open world nature of our specifications.% 

%% \sophia{Is is sentence superfluous now?.}
%% \sophia{In contrast to what we said on Friday's conf call we do not need to put any restrictions
%% on $\M'$ -- not even disjointness is required.}
%% \kjx{OK so in the \textbf{next iteration} we can just replace M;M' with a ' operator applied to any module\ldots}

The judgement $\M \mkpair  {\M'},\sigma \models \A$ means that  
assertion $\A$ is satisfied by  $\M \mkpair  {\M'}$ and $\sigma$.  
As in traditional specification languages \cite{Leavens-etal07,Meyer92}, satisfaction is judged 
in the context of a runtime configuration $\sigma$; but in addition, it is judged in the context of the internal and external modules.
%Satisfaction is also  judged in the context of modules; t
These are used to find   abstract functions defining ghost fields as well as  method bodies
needed when judging validity of temporal assertions such as
$\Future {\_}$. %: the modules contain the code necessary to reach those configurations.

\jm{The judgement $\M \mkpair  {\M'},\sigma_0 \ldots \sigma \models \A$ means that  
assertion $\A$ is satisfied by  $\M \mkpair  {\M'}$ and $\sigma$, with initial configuration $\sigma_0$.  
As in traditional specification languages \cite{Leavens-etal07,Meyer92}, satisfaction is judged 
in the context of a runtime configuration $\sigma$; but in addition, it is judged in the context of the internal and external modules.
%Satisfaction is also  judged in the context of modules; t
These are used to find   abstract functions defining ghost fields as well as  method bodies
needed when judging validity of temporal assertions such as
$\Future {\_}$.}\mrr[The initial configuration is provided to constrain the past execution to one possible path. Since the language \LangOO is deterministic, the initial state is sufficient to determine the exact steps which occurred. For the same reason, the future is also predetermined, keeping the temporal logic linear]{Added this further explanation. It otherwise becomes unclear why we need the initial configuration, and it can seem like "what am I missing", making it less accessible.}

We distinguish between internal and external modules. This %offers two advantages:
\sophia{has two uses:}
First, 
\Chainmail\ includes the ``$\External{\prg{o}}$'' assertion to require
that an object belongs to the external module, as in the Bank
Account's assertion (2) and (3) in
section~\ref{sect:motivate:Bank}. Second, we adopt a version of
visible states semantics \cite{MuellerPoetzsch-HeffterLeavens06,larch93,Meyer97}, treating all
executions within a module as atomic.
We only record runtime configurations which are {\em external}
 to module $\M$, \ie those where the
 executing object (\ie the current receiver) comes from module $\M'$.
 Execution % is  a judgment of 
 has the form\\
 $~ \strut  \hspace{1.3in}    \M \mkpair  {\M'},\sigma \leadsto \sigma'$\\  
% $\M \mkpair  {\M'},\sigma \leadsto \sigma'$\\  
where we ignore all intermediate steps
 with receivers  internal to $\M$. 
 % removed the below as it appears in next section.
%Similarly, when considering $\Arising {\M \mkpair  {\M'}}$, \ie the configurations arising from 
%executions in $\M \mkpair  {\M'}$, we can take method bodies defined in $\M$ or in $\M'$, but we will only consider the runtime 
%configurations which are external to $\M$.
%
In the next section we  shall 
outline the underlying programming language, and
define the judgment  $\M \mkpair  {\M'},\sigma \leadsto \sigma'$ and the set 
$\Arising {\M \mkpair  {\M'}}$ \jm{$\Arising {\M \mkpair  {\M'}, \sigma_0}$}.
 



