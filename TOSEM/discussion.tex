
\paragraph{Specification Language}

%\kjx{this section is new, but James didn't really know what Sophia wanted.}

The key to writing and using holistic specifications is to capture the
implicit assumptions underlying the design of the system we are
specifying.  The design principle animating the design of \Chainmail\
is to make writing holistic specifications as straightforward as
possible.

This is why we support bi-directional temporal operators (``will'',
``was'' etc) rather than just one temporal direction. Some assertions
are easier to express looking forwards (if a vending machine takes a
coin it will eventually dispense a chocolate bar) while other
assertions \jm{are} easier to express looking backwards (if a vending
machine dispenses a chocolate bar, it has already accepted the coin
to pay for it).

Similarly, in an imperative setting, the question of whether a change can
occur at all is at least as important as the precise details of the
change. This is why we support the ``changes'' operator in \Chainmail;
to enable specifications to capture that notion directly, rather than
e.g.\ expressing explicit differences in values between pre- and post-
states.

For modelling the open world, the interactions between the external
and internal modules are essential.
\sdf{Therefore in assertions we distinguish the internal from the external module,
$\M\mkpair \M', ... \models \A$,
and we consider that an internal module 
$\M$ satisfies an invariant $\A$, when all reachable states which lie outside $\M$ satisfy $\A$; 
for this, we introduced two-module execution $\M\mkpair \M',\sigma \leadsto \sigma'$.
 }
%\kjx{doesn't know what more to say here someone else can fill it in} 
We \sdf{also} included predicates $\External {\_}$ and     $\Internal {\_}$  to   %be able to state explicitly 
\sdf{state} % state is enough
which side of the boundary between modules \sdf{an object is}.
% SD chopped; nothing crosses, I think  and what can cross that boundary.


\sdf{A major aspect of the design of  \Chainmail was the treatment of recursion:
We wanted to support the writing of   predicates in
a natural way, and therefore allow for unrestricted recursion. However, unbounded recursion opens the possibility of infinite expansion, and
this, in turn may introduce contradictions into the logic.
Our solutions is based on a two-tier approach: We distinguish between expressions and assertions,
where expressions may appear in assertions, and where 
we allow unrestricted recursion in expressions.
Thus we do not support user-defined, recursive predicates, but do allow
\jm{user}-defined recursive functions in the form of ghost fields.
An assertion which consists of an expression has the value of that
expression, if the  expression has a  finite expansion returning  that a value,
and otherwise is $\false.$ 
Thus, expressions are not classical, but assertions are. 
For example, it does not hold for all boolean expressions $e$ that $\prg{if}\ \ \prg{e}\  \prg{then} \ \true \ \prg{else} \ \false$ returns $\true$, but it does hold  for all assertions $\A$ that $\A \vee \neg \A$ holds.
}


The issues we have had to deal with are those related to space in
various ways: ``access'', ``in'', and ``external''.  One object
referring to another appears simple enough on the surface, but causes
significant problems in an open world setting. In \Chainmail, these
problems appear with universal quantification on the source of a
reference: ``in'' enables us to bound these quantifications.



\paragraph{Underlying Language}

%\kjx{pulled in from earlier version}

For our underlying language, we have chosen an object-oriented, class based language,
%\sd{but we  expect that the ideas} will also be
%applicable to other kinds of languages (object-oriented or
%otherwise).
%% \sd{We  could, \eg} extend our work to prototype-based programming
%% by creating an (anonymous) class to reify each
%% prototype \cite{graceClasses}. 
%
We have chosen to use a dynamically typed language because many of the
problems we hope to address are written in these
languages: web apps and mashups in Javascript; backends in Ruby or
PHP.  We expect that supporting types would make the problem easier,
not harder, but at the cost of significantly increasing the complexity
of the trusted computing base that we assume will run our programs. In
an open world, without some level of assurance (e.g. proof-carrying
code) about the trustworthiness of type information: unfounded
assumptions about types can give rise to new vulnerabilities that
attackers can exploit \cite{pickles}. We don't address inheritance, as although it provides helpful abstractions, it doesn't change the fundamental problems, since in a dynamically-typed language it can be encoded through duplicating functionality in different classes.

As a specification language,
individual \Chainmail assertions can be combined or reused without any
inheritance mechanism: the semantics are simply that all
the \Chainmail\ assertions are expected to hold at all the points of
execution that they constrain.  \LangOO\ does not contain inheritance
simply because it is not necessary to demonstrate specifications of
robustness: whether an \LangOO class is defined in one place, or
whether it is split into many multiply-inherited superclasses, traits,
default methods in interfaces or protocols, etc.\ is irrelevant,
provided we can model the resulting (flattened) behaviour of such a
composition as a single logical \LangOO\ class.

\jm{
\paragraph{Linearity}
Earlier work on \Chainmail~\cite{FASE} featured a non-linear temporal logic
where all possible executions through a program were considered simultaneously.
In considering the past, all possible branches stemming from all possible 
initial configurations that might have preceded the current configuration
were considered. Similarly, when considering the future, all possible branches 
that might arise from the current configuration were considered. In this paper we
constrain satisfaction of temporal operators to consider only a single execution 
path at a time. This is done by (a) designing the underlying language such that its
execution is deterministic, thereby ensuring the future is linear, and  (b) considering 
satisfaction in relation to only a single initial configuration at a time, thereby 
ensuring the past is linear.
}

\jm{
A linear temporal logic provides \Chainmail with several useful properties that 
were previously not present. Lemma \ref{lemma:past_disj_distributive} states that 
disjunctions are distributive over the past.
\begin{lemma}[Past Distributivity]\label{lemma:past_disj_distributive} 
For assertions $\A_1$ and $\A_2$, 
disjunctions are distributive over the past.
\begin{enumerate}
\item
$\Prev{\A_1\ \vee\ \A_2}\ \equiv\ \Prev{\A_1}\ \vee\ \Prev{\A_2}$
\item
$\Past{\A_1\ \vee\ \A_2}\ \equiv\ \Past{\A_1}\ \vee\ \Past{\A_2}$
\end{enumerate}
\end{lemma}
\begin{proof}
Proofs of both  (1) and (2) follow similar strategies, and for simplicity we only show the 
proof of (1).
Since the past (both $\Prev{}$ and $\Past{}$) is restricted to  a single branch,
it follows that for all $\M_1$, $\M_2$, $\sigma_0$, and $\sigma$ such that 
$\M_1\ \mkpair\ \M_2, \sigma_0, \sigma\ \models\ \Prev{\A_1\ \vee\ \A_2}$ 
(or $\M_1\ \mkpair\ \M_2, \sigma_0, \sigma\ \models\ \Past{\A_1\ \vee\ \A_2}$),
there exists some $\sigma'$ such that 
$$
\M_1\ \mkpair\ \M_2 \bullet \sigma_0\ \leadsto^*\ \sigma'
$$
and
$$\M_1\ \mkpair\ \M_2 \bullet \sigma'\ \leadsto\ \sigma$$
and
$$\M_1\ \mkpair\ \M_2, \sigma_0, \sigma'\ \models\ \A_1\ \vee\ \A_2$$
It is then easy to demonstrate that either $\A_1$ or $\A_2$ must be 
satisfied by $\sigma'$, fulfilling the requirements to prove satisfaction of
$$\M_1\ \mkpair\ \M_2, \sigma_0, \sigma\ \models\ \Prev{\A_1}\ \vee\ \Prev{\A_2}$$
\end{proof}
It is perhaps not immediately clear why linearity is required for Lemma \ref{lemma:past_disj_distributive}.
If the past is not linear, then a configuration might have many previous configurations (in the case of $\Prev{}$),
each of which satisfy either $\A_1$ or $\A_2$, but some that don't satisfy one of $\A_1$ or $\A_2$.
}

\jm{
Lemma \ref{lemma:change_sat} states that if satisfaction differs over time, then 
there is a specific moment in time when that change occurs.
\begin{lemma}[Change of Satisfaction]\label{lemma:change_sat}
If satisfaction of an assertion changes over time, then there is a specific moment when that change occurs.
\begin{enumerate}
\item
$\A\ \wedge\ \Future{\neg \A}\ \subseteqq\ \Future{\Prev{\A}\ \wedge\ \neg \A}$
\item
$\A\ \wedge\ \Past{\neg \A}\ \subseteqq\ \Prev{\neg \A}\ \vee\ \Past{\Prev{\neg \A}\ \wedge\ \A}$
\end{enumerate}
\end{lemma}
\begin{proof}
We do not provide the complete proof, but rather a proof sketch.
The proof of both proceeds by induction on the pair reduction implied by $\Future{\neg \A}$ and $\Past{\neg \A}$, 
the base case being a single pair reduction step.
\end{proof}
Lemma \ref{lemma:change_sat} is particularly useful in constructing proofs of satisfaction 
in cases where quantities change over time. For example, the Bank Account example of 
Section \ref{sect:motivate:Bank} asserts that if the \prg{balance} changes, then it must happen 
as a result of a call to the \prg{deposit} method. Lemma \ref{lemma:change_sat} is required 
to identify the moment of change.}

\jm{
The difference between a linear and non-linear formulation of \Chainmail that grants the properties 
expressed in Lemmas \ref{lemma:past_disj_distributive} and \ref{lemma:change_sat} 
is in the semantics of the temporal operators. As an example, consider the result of Lemma \ref{lemma:change_sat}
in a non-linear \Chainmail where multiple possible branches are considered during satisfaction. 
While it is certainly true that if satisfaction of an assertion 
changes over time, then it follows that there is some moment when that change 
occurs, it does not mean that that change occurs at the same configuration in all
execution paths. Satisfaction of temporal operators does not put a path restriction
on future or past configurations. That is, when satisfaction of nested temporal operators 
is determined, satisfaction of operators at levels below the top level is not determined 
with respect to the path implied by the the top level. In Lemma \ref{lemma:change_sat}
the nested \prg{\Prev{}} considers all paths that might lead to the future (or past) 
configuration, and not just those implied by the top level \prg{\Future{}} (or \prg{\Past{}}).
}

\jm{
The \Chainmail presented in this paper is linear in part because of the non-determinism
of \LangOO, the underlying language.
Linearity is much easier to achieve if the underlying language is deterministic, 
as this means that the future is guaranteed to be linear, and only the initial configuration 
needs to be fixed. The obvious downside of this approach is that language semantics are 
not generally deterministic, and if the underlying language were not deterministic, 
the definition of satisfaction would have to be extended to account for a non-linear 
future. This could perhaps be done by dividing execution into discrete paths, and
determining satisfaction on a per-path basis, however this would likely be fairly complex.
Alternatively, the semantics of \Chainmail for a non-deterministic language might
consider branching execution, but include a mechanism for capturing the path context 
implied by temporal operators.
}


% Julian: below is some discussion about the coq proofs.

% SD removed, as some of this is in Related work
%\paragraph{Contracts and Preconditions}
%
%Traditional specification languages based on pre- and post-conditions
%are generally based on design-by-contract assumptions: ``if the
%precondition is not satisfied, the routine is not bound to do
%anything'' \cite{meyer92dbc} --- that is, the routine can do
%anything. Ensuring preconditions is the responsibility of the caller,
%and if the caller fails to meet that responsibility it is not the
%responsibility of the invoked function to fix the problem \cite{Mey88}.
%Underlying this approach, however, is the assumption of a closed
%system: that all the modules in a system are equally trusted, and so
%inserting redundant tests would just make the system more complex and
%more buggy. Indeed, \citet{meyer92dbc} states that ``This principle is
%the exact opposite of the idea of defensive programming.''
%
%In an open world, however, we do not have the luxury of trusting the
%other components with which we interact --- indeed we barely have the
%luxury of trusting ourselves.  We must assume that our methods may be
%invoked at any time, with any combination of arguments the underlying
%platform will support, irrespective of the overall state of the
%system, or of the object that receives the method request: in some
%sense, this is the very definition of an open system in an open world.
%Since methods cannot control when they are invoked, we must work
%as if all (potentially) externally-visible methods just have the precondition
%\prg{true} --- and we had better be very careful about any assumptions
%we make about which method or objects are in fact externally visible
%and which are not.  
%%\susan{I like it too.}\kjx{likes this sentence: 
%\sd{Holistic specifications directly support robust programming by making those kinds of
%assumptions explicit, giving the necessary conditions under which
%effects may take place and objects may become accessible, and then can
%help programmers ensure their programs maintain those conditions.}
%% WAS, but SD argues it is not right
%%Holistic specifications
%%directly support robust programming by making those kinds of
%%assumptions explicit, giving the necessary conditions under which
%%objects should be accessed or their methods invoked, and then can
%%help programmers ensure their programs maintain those conditions.}

 

%%% it's a point James would like made somewhre? but where?



%% \paragraph{Necessary v.s.\ sufficient conditions.}

%% Are the necessary conditions the same as the complement of all the
%% sufficient conditions?  The \sd{possible state transitions of a component are
%% described by the transitive closure of the individual transitions.
%% % , and
%% %the reachable states are those that are reachable from initial states through
%% %this transitive closure relation.
%% Taking  Figure \ref{fig:NecessaryAndSuff} part (a), if we calculate the 
%% complement of the transitive closure 
%% of the transitions, then we could, presumably, obtain all transitions which will not happen, 
%% and if we apply this relation to the set of initial states, we obtain all 
%% states guaranteed not to be reached.
%% Thus, using the sufficient conditions, we  obtain implicitly   
%% more fine-grained information than that  available through the yellow transitions and 
%% yellow boxes in Figure \ref{fig:NecessaryAndSuff}  part (b).}

%% \sd{This implicit approach} is mathematically sound. But it is impractical, brittle with
%% regards to software maintenance, and weak with regards to reasoning in
%% the open world.


%% \sd{The implicit approach} is impractical: it suggests that when interested in
%% a necessity guarantee a programmer would need to read the
%% specifications of all the functions in that module, \sd{and think 
%% about all possible sequences of such functions and all interleavings with 
%% other modules. }
%% What if the bank did indeed enforce that only the account owner may withdraw
%% funds, but had another function which allowed the manager to appoint
%% an account supervisor, and another which allowed the account
%% supervisor to assign owners?

%% \sd{The implicit approach} is also brittle with regards to software maintenance:
%%  it gives no guidance to the team maintaining a piece of
%% software.  If the necessary conditions are only implicit in the
%% sufficient conditions, then developers' intentions about those
%% conditions are not represented anywhere: there can be no distinction
%% between a condition that is accidental (if \sd{\prg{notify}-ing} is not
%% implemented, then it cannot be permitted) and one that is essential
%% (money can only be transferred by account owners).
%% Subsequent developers may inadvertently add functions which break these
%% intentions, without even knowing they've done so.

%% Finally, the implicit approach  is   weak when it comes to reasoning
%% about programs in an open world:  it does not give any
%% guarantees about objects when they are passed as arguments to calls
%% into unknown code. For example, what guarantees can we make about the
%% top of the DOM tree when we pass a wrapper pointing to lower parts of
%% the tree to an unknown advertiser?
