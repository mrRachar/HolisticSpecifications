%\section{The language \LangOO}
%\label{sect:LangOO}
\subsection{Modules and Classes}
\label{secONE}

\LangOO programs consist of modules, which are repositories of code. Since we study class based oo languages,
in this work, code is represented as classes, and  modules are mappings from  identifiers to class  descriptions.

\begin{definition}[Modules]
\label{defONE}
We define $\syntax{Module}$ as  the set of mappings from identifiers to class descriptions (the latter defined in Definition \ref{def:syntax:classes}):\\  % to force line break

\begin{tabular}  {@{}l@{\,}c@{\,}ll}
\syntax{Module} \ \  &  \   $\triangleq $  \ &
   $ \{ \ \ \M \ \ \mid \ \  \M: \ \prg{Identifier} \   \longrightarrow \
  \  \syntax{ClassDescr}     \  \    \}$
 \end{tabular}
\end{definition}
 
Classes, as defined   below,
consist of field, method definitions and ghost field declarations.
 \LangOO is untyped, and therefore fields are declared without types, 
 method signatures and ghost field signatures consist of  sequences of parameters without types, and no return type.
 Method bodies consist of sequences of statements;
these can be field read or field assignments, object creation, method calls, and return statements.
All else, \eg booleans, conditionals, loops,  can be encoded.
Field read or write is only allowed \sd{if the object whose field is being read 
belongs to the same class as the current method. This is enforced by the operational semantics, \cf
Fig.  \ref{fig:Execution}.}
\sd{Ghost fields  are defined as implicit, side-effect-free functions with zero or more parameters. They are ghost information, \ie 
they are not directly stored in the objects, and are not read/written during execution. When such a ghostfield is
mentioned in an assertion, the corresponding function is evaluated. More in section \ref{sect:expressions}.
Observe that the expressions that make up the bodies of ghostfield declarations (\prg{e}) are more complex than the terms that 
appear in individual statements.}

From now on we expect that the set of field and the set of ghostfields defined in a class are disjoint.  

\label{sec:syntax:classes}


\begin{definition}[Classes]
\label{def:syntax:classes}
Class descriptions consist of field declarations, method declarations, ghostfield declarations, and, optionally, a constructor.
 
\begin{tabular}{lcll}
 \syntax{ClassDescr}   &   \BBC  &     \kwN{class}  \syntax{ClassId} \lp \x$^*$\rp    \lb\,  $($ \syntax{FieldDecl} $)^*$ $($ CDecl\ $)?$ \
 $($  \syntax{MethDecl}\ $)^*$   \   $($   \syntax{GhosDecl}\ $)^*$ \ \ \rb
\\
\syntax{FieldDecl} &\BBC& \kwN{field} \f \\
\syntax{CDecl} &\BBC&
     \kwN{constructor}\    \lp \x$^*$\rp     \lb\, \syntax{Stmts}  \,
    \rb
 \\
\syntax{MethDecl} &\BBC&
     \kwN{method}\    \m\lp \x$^*$\rp     \lb\, \syntax{Stmts}  \,
    \rb
 \\
 \syntax{Stmts}  &\BBC&  \syntax{Stmt}     ~\SOR~  \syntax{Stmt} \semi \syntax{Stmts} \\
\syntax{Stmt}    &\BBC&
      \x.\f {\kw{:=}} \x   ~\SOR~  \x{\kw{:=}}  \x.\f    ~\SOR~        \x  {\kw{:=}} \x.\m\lp \x$^*$\rp     ~\SOR~     \x  {\kw{:=}}     \newKW\, \c\,\lp \x$^*$\rp   ~\SOR~
   \returnKW \,  \x   \\
  \syntax{GhostDecl} &\BBC&  \kwN{ghost} \f\lp \ \x$^*$\ \rp \lb \  \SE\ \rb\\
 \SE  &\BBC&    \kwN{true}   ~\SOR~  \kwN{false}   ~\SOR~  \kwN{null}  ~\SOR~  \x  \   ~\SOR~  
     \   \SE=\SE    ~\SOR~ \kwN{if}\, \SE\,   \kwN{then}\,  \SE\,    \kwN{else}\, \SE    ~\SOR~  \SE.\f\lp\ \SE$^*$ \ \rp\\
 \x, \f, \m &\BBC&  \prg{Identifier} 
 \end{tabular}

  \vspace{.03in}
  \noindent
 where we use metavariables as follows:
 $\x \in  \syntax{VarId} \ \ \  \f \in  \syntax{FldId} \ \ \  \m \in  \syntax{MethId} \ \ \  \c \in  \syntax{ClassId}$, and  \x\ includes \this
\end{definition}


We define a method lookup function $\mathcal{M}$, which returns the corresponding method definition given a class \c\ and a method identifier \m, and similarly a ghostfield lookup function $\mathcal{G}$, and a constructor lookup function $\mathcal{C}$, which returns a default, field-filling constructor if one wasn't defined.

 \begin{definition}[Lookup] For a class identifier \prg{C}  and a method identifier \prg{m}$:$  $ ~ $ \\
\label{def:lookup}
\noindent
$
\Meths {} {\prg{C}} {m}       \triangleq  \ \left\{
\begin{array}{l}
                        \m\, \lp \p_1, ... \p_n \rp \lb Stmts\, \rb\\
\hspace{0.5in} \mbox{if}\  \M(\prg{C}) =   \kwN{class}\, \prg{C}\, \  \lb ...\,   \kwN{method}\, \m\, \lp \p_1, ... \p_n \rp \lb Stmts\,  \rb  ... \ \rb.
\\
\mbox{undefined},  \ \ \ \mbox{otherwise.}
\end{array}
                    \right.$
\\
$
{\mathcal G} (\M, {\prg{C}}, {\f})    \ \   \triangleq  \ \left\{
\begin{array}{l}
                        \f\, \lp \p_1, ... \p_n \rp \lb \prg{e}\  \rb\\
\hspace{0.5in} \mbox{if}\  \M(\prg{C}) =   \kwN{class}\, \prg{C}\, \  \lb ...\,   \kwN{ghost}\,  \m\, \lp \p_1, ... \p_n \rp \lb \prg{e} \  \rb  ... \ \rb.
\\
\mbox{undefined},  \ \ \ \mbox{otherwise.}
\end{array}
                    \right.$
\\
$
{\mathcal C} (\M, {\prg{C}})   \quad \ \,   \triangleq  \ \left\{
\begin{array}{l}
                        \kwN{constructor} \, \lp \p_1, ... \p_n \rp \lb Stmts\, \rb\\
\hspace{0.5in} \mbox{if}\  \M(\prg{C}) =   \kwN{class}\, \prg{C}\, \  \lb ... \, \kwN{constructor} \, \lp \p_1, ... \p_n \rp \lb Stmts\, \rb ... \ \rb.
\\
\kwN{constructor} \, \lp \p_1, ... \p_n \rp \lb \prg{this}.\f_1 \kw{:=} \p_1; ... \prg{this}.\f_n \kw{:=} \p_n\ \rb\\
\hspace{0.5in} \mbox{where}\  \M(\prg{C}) =   \kwN{class}\, \prg{C}\, \  \lb \kwN{field}\, \f_1\, ... \kwN{field}\, \f_n\ ... \ \rb,  \ \ \ \mbox{otherwise.}
\end{array}
                    \right.$
  \end{definition}

\subsection{The Operational Semantics of \LangOO}
\label{formal:semantics}

We will now define execution of \LangOO code.
We start by  defining the  runtime entities, and runtime configurations, $\sigma$, which consist of heaps and stacks of frames.
 The frames are pairs consisting of a continuation, and a mapping from identifiers to values.
The continuation represents the code to be executed next, and the mapping gives meaning
to the formal and local parameters.

\begin{definition}[Runtime Entities]
\label{def:runtimeentities}
We define addresses, values, frames, stacks, heaps and runtime configurations.

\begin{itemize}
\item
We take addresses to be an  enumerable set,  \prg{Addr}, and use the identifier $\alpha\in \prg{Addr}$ to indicate an address.
\item
Values, $v$, are either addresses, or sets of addresses or null:\\
 $~ ~ ~ \ v \in \{ \prg{null} \} \cup \prg{Addr}\cup {\mathcal P}( \prg{Addr})$.
\item
  Continuations are either   statements  (as defined in Definition~\ref{def:syntax:classes}) or a marker, \x {\kw{:=}} $\bullet$, for a nested call followed by
  statements to be executed
  once the call returns.


\begin{tabular}{lcll}
\syntax{Continuation} &\BBC&   \syntax{Stmts} ~\SOR~   \x {\kw{:=}} $\bullet$ \semi\ \syntax{Stmts} \\
 \end{tabular}

\item
Frames, $\phi$, consist of a code stub  and a  mapping from identifiers to values:\\  $~ ~ ~ \ \phi \ \in\ \syntax{CodeStub} \times \prg{Ident} \rightarrow Value$,
\item
Stacks,  $\psi$, are sequences of frames, $\psi\ ::=   \phi \ | \ \phi\cdot\psi$.
\item
Objects consist of a class identifier, and a partial mapping from field identifier to values: \\  \ $~ ~ ~ \ Object\ = \ \prg{ClassID} \times (\prg{FieldId} \rightarrow Value)$.
\item
Heaps, $\chi$, are mappings from addresses to objects:\  \  $\chi\ \in\ \prg{Addr} \rightarrow Object$.
\item
Runtime configurations, $\sigma$, are pairs of stacks and heaps, $\sigma\ ::=\ (\ \psi, \chi\ )$.
\end{itemize}

\end{definition}


Note that values may be sets of addresses. Such values are never part of the execution of 
\LangOO, but are used to give semantics to assertions . %-- we shall see that in Definition \ref{def:valid:assertion}.
Next, we define the interpretation of variables (\x) and   field look up  (\x.\f)
in the context of frames,
heaps and runtime configurations; these interpretations are used to define the operational semantics and  also  the
validity of assertions, later on in Definitions 3-7. %HARD \ref{def:valid:assertion:space}:

\begin{definition}[Interpretations]
\label{def:interp}
We first define lookup of fields and classes, where $\alpha$ is an address, and \f\, is a field identifier:
\begin{itemize}
\item
$\chi ({\alpha},{\f})$ $\triangleq$  $\fldMap({\alpha},{\f})$\ \ \ if \ \ $\chi(\alpha)=(\_, \fldMap)$.
\item
$\ClassOf {\alpha} {\chi} $ $\triangleq$ $\c$\  \ \ if \ \ $\chi(\alpha)=(\c,\_)$
\end{itemize}

\noindent
We now define interpretations  as follows:

\begin{itemize}
\item
$\interp {\x}{\phi} $ $\triangleq$ $\phi(\x)$
\item
$\interp {\x.\f}{(\phi,\chi)} $ $\triangleq$ $v$, \ \ \ if \ \ $\chi(\phi(\x))=(\_, \fldMap)$ and $\fldMap(\f)$=$v$

\end{itemize}

\noindent
For ease of notation, we also use the shorthands below:
\begin{itemize}
\item
$\interp {\x}{(\phi\cdot\psi,\chi)} $ $\triangleq$ $\interp {\x}{\phi} $
\item
$\interp {\x.\f}{(\phi\cdot\psi,\chi)} $ $\triangleq$ $\interp  {\x.\f}{(\phi,\chi)} $
\item
$\ClassOf {\alpha} {(\psi,\chi)} $ $\triangleq$ $\ClassOf {\alpha} {\chi} $
\item
$\ClassOf {\x} {\sigma} $ $\triangleq$ $\ClassOf {\interp {\x}{\sigma}} {\sigma} $
\end{itemize}

\end{definition}

In the definition of the operational semantics of \LangOO we use the following notations for lookup and updates of runtime entities :

\begin{definition}[Lookup and update of runtime configurations]
We define convenient shorthands for looking up in  runtime entities.
\begin{itemize}
\item
Assuming that $\phi$ is the tuple  $(\prg{stub}, varMap)$, we use the notation  $\phi.\prg{contn}$ to obtain \prg{stub}.
\item
Assuming a value v, and that $\phi$ is the tuple  $(\prg{stub}, varMap)$, we define $\phi[\prg{contn}\mapsto\prg{stub'}]$ for updating the stub, \ie
$(\prg{stub'}, varMap)$.   We use  $\phi[\x \mapsto v]$  for updating the variable map, \ie  $(\prg{stub}, varMap[\x \mapsto v])$.
\item
Assuming a heap $\chi$, a value $v$, and   that $\chi(\alpha)=(\c, fieldMap)$,
we use $\chi[\alpha,\f \mapsto v]$ as a shorthand for updating the object, \ie $\chi[\alpha \mapsto (\c, fieldMap[\f \mapsto v]]$.
\end{itemize}

\end{definition}



\begin{figure*}
$\begin{array}{l}
\inferenceruleNN {methCall\_OS} {
\\
\phi.\prg{contn}\ =\ \x {\kw{:= }} \x_0.\m \lp \x_1, ... \x_n \rp \semi \prg{Stmts}
\hspace{2cm}
\interp{\x_0}{\phi} = \alpha
\\
\Meths {} {\ClassOf {\alpha} {\chi}} {\m} \  =  \ \m\lp \p_1, \ldots \p_n \rp \lb \prg{Stmts}_1 \,  \rb
  \\
 \phi''\ =\  (\  \prg{Stmts}_1,\ \ (\ \this \mapsto \alpha,
  \p_1 \mapsto  \interp{\x_1}{\phi}, \ldots \p_n \mapsto  \interp{\x_n}{\phi}\ ) \ )
}
{
 \M,\, (\ \phi\cdot\psi,\ \chi\ )\ \ \leadsto\  \ (\ \phi''\cdot\phi[\prg{contn}\mapsto\x  \kw{:=} \bullet \semi \prg{Stmts}] \cdot\psi,\ \chi\ )
}

\\ \\
\inferenceruleNN {varAssgn\_OS} {
 \phi.\prg{contn} \ = \ \x  {\kw{:= }}  \y.\f \ \semi \prg{Stmts}\ \hspace{2cm} \ClassOf {\y} {\sigma} =\ClassOf {\this} {\sigma}
}
{
 \M,\,  (\ \phi\cdot\psi, \chi\ )\ \ \leadsto\  \ (\ \phi[ \prg{contn} \mapsto \prg{Stmts}, \x\mapsto \interp{\y.\f}{\phi,\chi}] \cdot\psi,\ \chi\  )
}
\\
\\
\inferenceruleNN{fieldAssgn\_OS} {
 \phi.\prg{contn}\ =\  \x.\f  \kw{:=} \y  \semi \prg{Stmts} \hspace{2cm} \ClassOf {\x} {\sigma} =\ClassOf {\this} {\sigma}
}
{
 \M,\,  (\ \phi\cdot\psi, \chi\  )\ \ \leadsto\  \ (\ \phi[\prg{contn}\mapsto  \prg{Stmts} ] \cdot\psi, \chi[\interp{\x}{\phi},\f \mapsto \interp{\y}{\phi,\chi}]\  )
}
\\
\\
\inferenceruleNN {objCreate\_OS} {
 \phi.\prg{contn}\ =\  \x  \kw{:=} \kwN{new }\, \c \lp \x_1, ... \x_n \rp  \semi \prg{Stmts}
 \hspace{2cm}
 \alpha\ \mbox{new in}\ \chi
 \\
{\mathcal C}(\M, \c) = \kwN{constructor} \lp \p_1, \ldots \p_n \rp \lb \prg{Stmts}_1\,\rb

  \\
 \phi''\ =\  (\  \prg{Stmts}_1\semi \kwN{return}\ \prg{this},\ \ (\ \this \mapsto \alpha,
  \p_1 \mapsto  \interp{\x_1}{\phi}, \ldots \p_n \mapsto  \interp{\x_n}{\phi}\ ) \ )

}
{
 \M,\,  (\ \phi\cdot\psi, \chi\ )\ \ \leadsto\  \ (\ \phi''\cdot \phi[\prg{contn}\mapsto  \x  \kw{:=} \bullet \semi\prg{Stmts}] \cdot\psi, \ \chi[\alpha \mapsto (\c, \emptyset  ) ]\ )
}
\\
\\
\inferenceruleNN {return\_OS} {
 \phi.\prg{contn}\ =\   {\kwN{return }}\, \x  \semi \prg{Stmts}\ \  \ or\  \ \  \phi.\prg{contn}\ =\   {\kwN{return}}\, \x
 \\
\phi'.\prg{contn}\ =\  \x' \kw{:=} \bullet  \semi \prg{Stmts}'
}
{
 \M,\,  (\ \phi\cdot\phi'\cdot\psi, \chi\ )\ \ \leadsto\  \ (\ \phi'[\prg{contn}\mapsto  \prg{Stmts'},\x' \mapsto \interp{\x}{\phi}] \cdot\psi, \ \chi \ )
}
\end{array}
$
\caption{Operational Semantics}
\label{fig:Execution}
\end{figure*}

Execution of a statement has the form $\M, \sigma \leadsto \sigma'$, and is defined in figure \ref{fig:Execution}.

\begin{definition}[Execution] of one or more steps is defined as follows:

\begin{itemize}
     \item
   The relation $\M, \sigma \leadsto \sigma'$, it is defined in Figure~\ref{fig:Execution}.

   \item
   $\M, \sigma \leadsto^* \sigma'$ holds, if i) $\sigma$=$\sigma'$, or ii) there exists a $\sigma''$ such that
   $\M, \sigma \leadsto^* \sigma''$ and $\M, \sigma'' \leadsto \sigma'$.
 \end{itemize}

\end{definition}
 
 % SD: I hid this section as it looks as if we are not iusing it sany more
%\subsection{Definedness of execution, and extending configurations}
%
%Note that interpretations and executions need not always be defined.
%For example, in a configuration whose top frame does not contain \x\,  in its domain, $\interp {\x}{\phi} $ is undefined. We define the relation $\sigma \subconf \sigma'$ to express that   $\sigma$ has more information than $\sigma'$, and then prove that more defined configurations preserve interpretations:
%
%\begin{definition}[Extending runtime configurations]
%The relation $\subconf$   is defined on runtime configurations as follows. Take arbitrary
%configurations $\sigma$, $\sigma'$, $\sigma''$, frame $\phi$, stacks $\psi$, $\psi'$,  heap $\chi$, address $\alpha$ free in $\chi$, value $v$ and object $o$, and define $\sigma  \subconf \sigma'$ as the smallest relation such that:
%
%\begin{itemize}
%\item
%$\sigma  \subconf \sigma$
%\item
%$(\phi[\x \mapsto v]\cdot \psi, \chi) \subconf  (\phi\cdot \psi, \chi)$
%\item
%$(\phi\cdot\psi\cdot\psi', \chi) \subconf  (\phi\cdot \psi, \chi)$
%\item
%$(\phi, \chi[\alpha \mapsto o) \subconf  (\phi\cdot \psi, \chi)$
%\item
%$\sigma'  \subconf \sigma''$ and $\sigma''  \subconf \sigma$ imply $\sigma'  \subconf \sigma$
%\end{itemize}
%\end{definition}
%
%
%
%\begin{lemma}[Preservation of interpretations and executions]
%If $\sigma'  \subconf \sigma$, then
%
%\begin{itemize}
%\item
%If $\interp {\x}{\sigma}$ is defined,\ \  then $\interp {\x}{\sigma'}$=$\interp {\x}{\sigma}$.
%\item
%If $\interp {\this.\f}{\sigma}$ is defined,\ \  then $\interp {\this.\f}{\sigma'}$=$\interp {\this.\f}{\sigma}$.
%\item
%If $\ClassOf {\alpha} {\sigma} $  is defined, \ \ then  $\ClassOf {\alpha} {\sigma'} $  = $\ClassOf {\alpha} {\sigma} $.
%\item
%If $\M, \sigma \, \leadsto^*\, \sigma''$, \ \  then     \ \ there exists a $\sigma''$, so that\ $\M, \sigma'\, \leadsto^*\, \sigma'''$
%and $\sigma''' \subconf \sigma''$.
%\end{itemize}
%\end{lemma}
%
%Note however, that such preservation does not hold for assertion. For example, if $\sigma'  \subconf \sigma$ , then 
%$\M \mkpair \M',\sigma \models \forall x.\A$ and does not imply $\M
%\mkpair \M',\sigma' \models \forall x.\A$; on the other hand, 
% $\M \mkpair \M',\sigma' \models \exists x.\A$  does not imply $\M \mkpair \M',\sigma \models \exists x.\A$
%

\Mrr[Moved this here, I'm still not sure if this is the kind of language we want for this, or the exact placement, but it sort of works]{Note that previous work on \Chainmail omitted constructors entirely as a simplification, however this prevented control over fields' initial values.  This required invariants over the fields to be qualified with a predicate to check the instance was constructed properly, or had existed in a correct state, complicating practical examples.}

\subsection{Module linking}

When studying validity of assertions in the open world we are concerned with whether   the  module
under consideration makes  a certain guarantee when executed in conjunction with other modules. To answer this, we
 need the concept of linking other modules to the module  under consideration.
 Linking, $\link$ ,  is an operation that takes two modules, and creates a module which corresponds  to the union of the two.
 %We use the concept of module linking in order to model the open world, where our module $\M$ whose code we know, will be executed together with further modules whose code we do not know.
We place some conditions for module linking to be defined: We require that the two modules do not contain implementations for the same class identifiers,

%SD removed the below as I think it is settled.
%\susan{where does the aux come from? I think what you said in the fragment calculus about disjointedness is neater} 
%\sophia{aux is defined in last line of Def. below. In the Frag Calculus the modules were not mappings, so we did not need something like aux; any idea how to avoid?}


\begin{definition}[Module Linking]
\label{def:link}
The linking operator\  \ $\link:\  \syntax{Module} \times  \syntax{Module} \longrightarrow \syntax{Module}$ is defined as follows:

$
\M \link \M{'}  \ \triangleq  \ \ \left\{
\begin{array}{l}
                        \M\ \link\!_{aux}\ \M{'},\ \ \   \hbox{if}\  \ dom(\M)\!\cap\!dom(\M')\!=\!\emptyset\\
\mbox{undefined}  \ \ \ \mbox{otherwise.}
\end{array}
                    \right.$

and where,
\begin{itemize}
     \item
   For all  $\prg{C}$: \ \
   $(\M\ \link\!_{aux}\ \M')(\prg{C})\  \triangleq  \ \M(\prg{C})$  if  $\prg{C}\in dom(\M)$, and  $\M'(\prg{C})$ otherwise.
 \end{itemize}
\end{definition}

Some properties of linking are described in lemma \ref{lemma:linking} in the main text. \sophia{
%The below says  that linking is associative and commutative, and preserves execution.
%
%\begin{lemma}[Properties of linking -- repetition of Lemma 1 in the main text] %HARD
%%\ref{lemma:linking} in the main text]
% For any modules $\M$,   $\M'$ and $\M''$, and runtime configurations $\sigma$, and $\sigma'$ we have$:$
%
%
% \begin{enumerate}
%     \item
%     $(\M \link \M')\link \M''$ = $\M \link (\M' \link \M'')$.
%    \item
%      $\M \link \M'$  = $\M' \link\M$.
%      \item
%      $\M, \sigma \leadsto \sigma'$, and $\M\link \M'$ is defined, \  \  implies\ \   $\M\link \M', \sigma \leadsto \sigma'$
%   \end{enumerate}
%
% \end{lemma}
 For the proof, 
% \begin{proof}
\sd{ (1) and (2) follow from Definition \ref{def:link}. (3) follows from \ref{def:link}, and the fact that if a lookup $\mathcal \M$ is
defined for $\M$, then it is also defined for $\M\link\M'$ and returns the same method, and similar result for class lookup.}
%  \end{proof}
}


 \subsection{Module pairs and visible states semantics}

A module $\M$ adheres to an invariant assertion  $\A$, if it satisfies
$\A$ in all runtime configurations that  can be reached through execution of the code of $\M$ when linked to that
of {\em any other} module $\M'$, and
which are {\em external} to $\M$. We call external to $\M$ those
configurations which are currently executing code which does not come from $\M$. This allows the code in $\M$ to break
the invariant internally and temporarily, provided that the invariant is observed across the states visible to the external client $\M'$.

We have defined two module execution in the main paper, Def. \ref{def:execution:internal:external}.
%Therefore, we define execution in terms of an internal module $\M$ and an external module $\M'$, through the judgment $\M \mkpair \M', \sigma \leadsto \sigma'$, which mandates that $\sigma$ and $\sigma'$ are external to $\M$, and that there exists an execution which leads from $\sigma$ to $\sigma'$ which leads through intermediate configurations
%$\sigma_2$, ...  $\sigma_{n+1}$ which are all internal to $\M$, and thus unobservable from the client.
%In a sense, we "pretend" that all calls to functions from $\M$ are executed atomically, even if they involve several intermediate,
%internal steps.
%
%
%\begin{definition} [Repeating definition 2 from main text] %HARD \ref{def:execution:internal:external}]
%Given runtime configurations $\sigma$,  $\sigma'$,  and a module-pair $\M \mkpair \M'$ we define
%execution where $\M$ is the internal, and $\M'$ is the external module as below:
%
%\begin{itemize}
%\item
%$\M \mkpair \M', \sigma \leadsto \sigma'$ \IFF
%there exist  $n\geq 2$ and runtime configurations $\sigma_1$,  ...
%$\sigma_n$, such that
%\begin{itemize}
%\item
%$\sigma$=$\sigma_1$,\ \  \ \ and\ \ \ \ $\sigma_n=\sigma'$.
%\item
%$\M \link \M', \sigma_i \leadsto \sigma_{i+1}'$,\  \  for $1\leq i \leq n\!-\!1$
%\item
%$\ClassOf{\interp {\this} {\sigma}} {\sigma}\not\in dom({\M})$,  \ \  \ \ and\ \ \ \
%$\ClassOf{\interp {\this} {\sigma'}} {\sigma'} \not\in dom({\M})$,
%\item
% $\ClassOf{\interp {\this} {\sigma_i}} {\sigma_i} \in dom({\M})$,\ \ \ \ for $2\leq i \leq n\!-\!2$
%\end{itemize}
%\end{itemize}
%
%\end{definition}

In \sophia{that} definition % above 
$n$ is allowed to have the value $2$. In this case the final bullet is trivial and  there exists a direct, external transition from $\sigma$ to $\sigma'$.  Our definition is related to the concept of visible states semantics, but differs in that visible states semantics select the configurations at which an invariant is expected to hold, while we select the states which are considered for executions which are expected to satisfy an invariant. Our assertions can talk about several states (through the use of the $\Future {\_}$ and $\Past{\_}$ connectives), and thus, the intention of ignoring some intermediate configurations can only be achieved if we refine the concept of execution. 

% SD chopped, as too similar to main text
% The following lemma states that linking external modules preserves execution
%
%\begin{lemma}[Linking modules preserves execution]
%\label{lemma:module_pair_execution}
%For any modules $\M$, $\M'$, and $\M''$, whose domains are pairwise disjoint, and runtime configurations $\sigma$, $\sigma'$,
%
%\begin{itemize}
%\item
% $\M \mkpair \M', \sigma \leadsto \sigma'$  implies $\M \mkpair (\M'\link\M'') ,\sigma \leadsto \sigma'$.  
%\item
%  $\M \mkpair \M', \sigma \leadsto \sigma'$  implies
%$(\M\link\M'') \mkpair \M' , \sigma \leadsto \sigma'$.
%
%\end{itemize}
%\end{lemma}
%
%\begin{proof} For the second guarantee  we use the fact that   $\M \mkpair \M', \sigma \leadsto \sigma'$ implies that all
%intermediate configurations are internal to $\M$ and thus also to $\M\link\M''$.
%\end{proof}

\sophia{We have defined initial and arising configurations  in Definition \ref{def:arise}. Note that there are infinitely many different initial configurations, they will be differing in the code stored in the continuation of the unique frame.}

%
%We can now answer the question as to which runtime configurations are pertinent when judging a module's
%adherence to an assertion.
%First, where does execution start? We define {\em initial} configurations to be those which may contain arbitrary code stubs, but which contain no objects. Objects will be created, and further methods will be called through execution of the code in $\phi.\prg{contn}$. From such initial configurations, executions of code from $\M \mkpair \M'$ creates a set of {\em arising} configurations, which, as we will see in Definition \ref{def:module_satisfies}, are pertinent when judging $\M$'s  adherence to assertions.
%
%\begin{definition}[Initial and arising Configurations -- repeating Definition \ref{def:arise}] are defined as follows: \label{defn:iniial-and-arising}
%
%\begin{itemize}
%     \item
%   $\Initial {(\psi,\chi)}$, \ \ if \ \ $\psi$ consists of a single frame $\phi$ with $dom(\phi)=\{ \this \}$, and there exists  some address $\alpha$, such that \ \ \    $\interp {\this}{\phi}$=$\alpha$, and \ $dom(\chi)$=$\alpha$,\  and\  
%    $\chi(\alpha)=(\prg{Object},\emptyset)$.
% \item
% $\Arising  {\M\mkpair\M'} \ = \ \{ \ \sigma \ \mid \ \exists \sigma_0. \ [\  \Initial{\sigma_0} \  \ \wedge\ \  \M\mkpair\M', \sigma_0 \leadsto^* \sigma \ \ ] \ \ \} $
% \end{itemize}
%
%\end{definition}


