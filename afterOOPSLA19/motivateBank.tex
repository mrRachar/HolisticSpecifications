

As a motivating example, we consider a simplified banking application,
with objects representing \prg{Account}s or \prg{Bank}s. 
As in \cite{ELang},   \prg{Account}s belong to \prg{Bank}s and hold money (\prg{balance}s);  
with access  to two \prg{Account}s of the same  \prg{Bank} one can  transfer any amount of money from
 one to the other.  We give a traditional specification in Figure \ref{fig:BankSpec}.

%SD changed function to method, to fir what comes later.
\begin{figure}[htbp] 
\begin{lstlisting}
   method deposit(src, amt)
   PRE:  this,src:Account $\wedge$  this$\neq$src $\wedge$ this.myBank=src.myBank $\wedge$ 
         amt:$\mathbb{N}$  $\wedge$   src.balance$\geq$amt
   POST: src.balance=src.balance$\pre$-amt $\wedge$ this.balance=this.balance$\pre$+amt

   method makeNewAccount(amt)
   PRE: this:Account $\wedge$  amt:$\mathbb{N}$ $\wedge$  this.balance$\geq$amt
   POST: this.balance=this.balance$\pre$-amt $\wedge$ fresh result $\wedge$ 
         result: Account $\wedge$ this.myBank=result.myBank $\wedge$ result.balance=amt

   method newAccount(amt)
   PRE:  this:Bank  
   POST:  result: Account $\wedge$  result.myBank=this $\wedge$ result.balance=amt
 \end{lstlisting}
 \vspace{-.8cm}
\caption{Functional specification of \prg{Bank} and \prg{Account}
%
}
\label{fig:functionalSpecBankAccount}
\label{fig:BankSpec}
\end{figure} 

The PRE-condition of \prg{deposit} requires that  the receiver and the
first argument  (\prg{this} and \prg{src}) are \prg{Account}s
and belong to the same bank,
that the second argument (\prg{amt}) is a number, and that \prg{src}'s
balance is at least \prg{amt}.
The POST-condition mandates that \prg{amt} has been transferred from \prg{src} to the receiver.
 The function \prg{makeNewAccount}  returns a fresh \prg{Account} with the same bank, and transfers \prg{amt}
 from the receiver \prg{Account} to the new \prg{Account}.
 Finally, the function \prg{newAccount} when run by a \prg{Bank} creates a new \prg{Account} with corresponding 
 amount of money in it.\footnote{{Note that our very limited bank specification doesn't even have the concept of an account owner.}}
% \se{As specified in our \prg{BankSpec} the only way to put money into the \prg{Bank} is with a call to \prg{newAccount} and there is no way to remove money from the \prg{Bank}.}\sophia{We can say that, but is it important? We do not even use "newAccount" anywhere}

\forget{
\emph{Aside} Notice that the specification means that access to an \prg{Account} allows anyone to withdraw all the money it holds
-- the concept of account owner who has exclusive right of withdrawal is not supported.
This simplified view allows  us to keep the example short, but compare with appendix \sophia{TO ADD}
for a specification which supports owners. \emph{end aside}\sophia{which is the best place to say that?}
}

With such a specification %it is enough%to let us
%calculate the result of operations on the accounts ---
%for example % it is straightforward to determine that 
the code below  satisfies its assertion. Assume that 
\prg{acm\_acc} and \prg{auth\_acc} are \prg{Account}s for the ACM and
for a conference paper author respectively. The ACM's \prg{acm\_acc}
has a balance of 10,000 before an author is 
registered, while afterwards it has a balance of 11,000. Meanwhile the
\prg{auth\_acc}'s balance  will be 500 from a starting balance of 1,500
(barely enough to buy a round of drinks at the conference hotel bar).

\begin{lstlisting}
  assume acm_acc,auth_acc: Account $\wedge$ acm_acc.balance=10000 $\wedge$  auth_acc.balance=1500
  acm_acc.deposit(auth_acc,1000)
  assert acm_acc.balance=11000  $\wedge$ auth_acc.balance=500
\end{lstlisting}

\vspace{-.2in}

This reasoning is fine in a closed world, where we only have to
consider complete programs, where all the code in our programs (or any
other systems with which they interact) is under our control.   
In an
open world, however, things are more complex: our systems will be made
up of a range of 
%modules, 
components, many of which we do not control; and
furthermore will have to interact with external systems which we
certainly do not control.  Returning to our author, say some time
after registering by executing the \prg{deposit} code above, they
attempt to pay for a round at the bar.  Under what circumstances can
they be sure they have enough funds in their account?

To see the problem, what if the bank provided a \prg{steal} method that 
 emptied out every account in the bank into a thief's account.
 %\se{(also in the \prg{Bank})}?\sophia{yes, also in the bank, but does ti matter? I want to avoud having too many facets in the story.}
%all their funds into the thief's account.
%consider the additional function specified below.
% The bank additionally provides a
%\prg{steal} method that empties out every account in the bank and puts
%all their funds into the thief's account. 
If this method existed and
if it were somehow called between registering at the conference and
going to the bar, then the author 
%(actually everyone using the same bank)
%would find an empty account (as would every other account holder other than the thief, of course).
would find an empty account.
%\susan{I have removed the statements that implied people had accounts because we don't have that in the spec}
%

%\vspace{-2in}
%\begin{lstlisting}
%   function steal(thief)
%   PRE:  this:Bank $\wedge$  thief.myBank=this 
%   POST: $\forall$a:Account.[a.myBank=this $\rightarrow$ a.balance=0 ] $\wedge$ 
%         thief.balance=thief.balance$\pre$+$\sum_{\prg{a}:\prg{Account} \wedge \prg{a.myBank}=\prg{this}}$ a.balance$\pre$
% \end{lstlisting}
%\vspace{-2in}
 
The critical problem is that a bank implementation including a \prg{steal}
method would meet the functional specification of the bank from
fig.~\ref{fig:functionalSpecBankAccount}, so long as the methods \prg{deposit},
\prg{makeNewAccount}, and \prg{newAccount}    meet
their specification.

One obvious solution would be to return to a closed-world
interpretation of specifications: we interpret specifications such as
fig.~\ref{fig:BankSpec} as \emph{exact} in the sense that only
implementations that meet the functional specification exactly,
\emph{with no extra methods or behaviour}, are considered as suitable
implementations of the functional specification. The problem is that
this solution is far too strong: it would for example rule out a bank
that  during software maintenance was given a new method \prg{count}
that simply counted the number of deposits that had taken place, or a method \prg{notify}
to enable the bank to occasionally send notifications  to its customers.
% occasionally paid interest to its clients.
%i.e. met fig.~\ref{fig:extras} as well as fig.~\ref{fig:BankSpec}.
%
%\begin{figure}[tbp]
%\begin{lstlisting}
%   function deposit(src, amt)
%   PRE:  $\mbox{ ... as before ...}$  
%   POST:  $\mbox{ ... as before ...}$  this.myBank.count=this.myBank.count$\pre$+1
% \end{lstlisting}
%\vspace{-.8cm}
%\caption{Counting deposits} %and paying interest}
%\label{fig:extras}
%\end{figure} 
% removed as payInterest affects the balance
%
%   function payInterest()
%   PRE: this:Bank  
%   POST: $\forall$a:Account.[a.myBank=this $\rightarrow$ a.balance=a.balance$\pre$+a.balance$\pre$/1000]


What we need is some way to permit bank implementations that 
%meet fig.~\ref{fig:extral} 
send notifications to customers, but to forbid implementations of \prg{steal}. % that meet fig.~\ref{fig:steal}.  
The key here is to capture the (implicit)
assumptions underlying fig.~\ref{fig:BankSpec}, and to provide
additional specifications that capture those assumptions.  The following
 three informal requirements   prevent methods like \prg{steal}:

\begin{enumerate}
\item % After creation,
% it said
%  the \emph{only} way and account -- SD chnaged as I thibk ti has slifghtly wrong meaning 
An account's
  balance can be changed only  if a client   calls the \prg{deposit} method  with the
  account as the receiver or as an argument. 
 % \sophia{I changed the wording as I want to show "change -> call to deposit"}
%  \susan{point out that this is not sufficient to stop steal because it could be rewritten to use deposit.}
%  \sophia{I think that is what we say below}
\item An account's balance can be changed  only  if a client has
  access to that  particular account. % object.
  % I want the structure of (1) and (2) to be similar.
\item The \prg{Bank}/\prg{Account} component does not leak access to existing accounts or banks. 
%SD I chnaged the belwo
%Neither accounts nor banks can leak money.   \susan{ I changed what you had in point 3 because it didn't make sense in context. I don't think what I put is what you were saying is it? Did you want to say something like 'There is no way that references to either accounts or banks can be passed to third parties?}
\end{enumerate}

Compared with the functional specification we have seen so far, these
requirements %assumptions 
capture \emph{necessary} %conditions 
rather than
\emph{sufficient} conditions:  Calling the \prg{deposit}
method to gain access to an account %is called to change an account's balance, and it is necessary
is necessary for any change to that account taking place.
%that the particular account object can be passed as a parameter to
%that method. 
The  function 
\prg{steal} is inconsistent with requirement  (1), as it reduces the balance of an \prg{Account} without calling the
function \prg{deposit}. 
However, requirement  (1) is not enough to protect our money. We need to (2) to avoid an \prg{Account}'s balance getting
modified without access to the particular \prg{Account}, and (3) to ensure that such accesses are not leaked. 

%\james{NEED TO DECIDE ON CONTRIBUTIONS.
%The contribution of this paper is a specification langauge and
%semantics that can be used to specify necessary specifications, and a
%semantics for those specifications that can determine whether some
%functional (sufficient) specifications are consistent (or not) with
%the necessary specifications. Also think that we do not do
%"can determine whether some
%functional (sufficient) specifications are consistent (or not) with
%the necessary specifications."}

 
We can  express these  requirements  %assumptions using 
through \Chainmail assertions.  Rather than %specifying \prg{functions} to  describe
specifying the behaviour of particular methods when they are called, we
write  assertions   that range across the entire behaviour of the
\prg{Bank}/\prg{Account}  module.%\james{note (1) and (2) use changes}
\vspace{.2cm}

%\begin{figure}[htbp]
%%\begin{definition}
%\label{def:pol2}

(1)\ \  $\triangleq$\ \ $\forall \prg{a}.[\ \ \prg{a}:\prg{Account} \wedge \Changes{\prg{a.balance}}  \ \    
    \longrightarrow \ \    \hfill$ \\
  $\strut \hspace{2.3cm} 
% TODO explain:
% we no longer need Past here, as we are ion visible states 
  \exists \prg{o}. [\    \Calls{\prg{o}}{\prg{deposit}}{\prg{a}}{\_,\_} \vee\  \Calls{\prg{o}}{\prg{deposit}}{\_}{\prg{a},\_}\  \ ] \ \ \ \ ] \hfill $

\vspace{.4cm}

    (2)\ \  $\triangleq$\ \ $\forall \prg{a}.\forall \prg{S}:\prg{Set}.\ [  \ \  \prg{a}:\prg{Account}\ \wedge \  \Using{ \Future{\,\Changes{\prg{a.balance}}} \,}{\prg{S}\,}\ \ \   \
    \longrightarrow$ \\
 $\strut \hspace{3.9cm} \hfill \exists \prg{o}.\ [\, \prg{o}\in \prg{S}\ \wedge\  \External{\prg{o}}  \ \wedge \ \CanAccess{\prg{o}}{\prg{a}} \, ] \ \ \ \ ]$
\vspace{.4cm} 
 
     (3)\ \  $\triangleq$\ \ $\forall \prg{a}.\forall \prg{S}:\prg{Set}.\ [ \ \  \prg{a}:\prg{Account}\ \wedge \ \ {\Using{\Future{\ \exists \prg{o}.[\ \External{\prg{o}} \ \wedge\ \CanAccess{\prg{o}}{\prg{a}}]}}{\SF}}$ \\  
 $\strut \hspace{3.9cm} \hfill   \longrightarrow \ \ \ \exists \prg{o}'.\ [\, \prg{o}'\in \prg{S}\  \wedge  \ \External{\prg{o}'}  \ \wedge \ \CanAccess{\prg{o}'}{\prg{a}}   \ ] \ \ \ \ ]$

%\end{definition}
%\vspace{-.2cm}
%\caption{Necessary specifications for \prg{deposit}}
%%\label{fig:nec}
% \end{figure}
\vspace{.2cm}

%% We will discuss the meaning of  \Chainmail assertions in more detail in the next sections, but 
%% give here a first impression, and discuss (1)-(3):
\noindent 
In the above and throughout the paper, we use an underscore ($\_$) to indicate an existentially bound variable whose 
value is of no interest.

\vspace{.2cm}

Assertion (1) %in fig.~\ref{fig:nec} says 
says that if   an account's balance changes
($\Changes{\prg{a.balance}}$),
then there must be some client object \prg{o}
that % in the past (${\Past{...}}$) 
called the \prg{deposit} method with \prg{a} as a receiver or as an argument 
($\Calls{\prg{o}} {\prg{deposit}} {\_} {\_}$).
%\sophia{ explain in some later section on that  this assertion implicilty gives   that \prg{o} is external}
%bank and its associated acc    ounts ($\prg{o} \notin\prg{Internal}(\prg{a})$)" -- and I have chopped it from
%the spec. This is because of the visible states :-)}
 
Assertion (2) similarly constrains any possible change to an 
account's balance: 
If at some future point the balance changes  (${\Future{\,\Changes{...}}}$),  % (${\Future{...}}$) %
and if this future change is observed with the state restricted to the objects from \SF~ (\ie $\Using{...}{\prg{S}}$), then 
at least one of these objects ($\prg{o}\in\SF$) is external to the \prg{Bank}/\prg{Account} system ($\External{\prg{o}}$) and 
has (direct) access to that account object
($\CanAccess{\prg{o}}{\prg{a}}$).
Notice that while the change in the \prg{balance} happens some time in the future,
the external object \prg{o} has access to \prg{a} in the \emph{current} state.
Notice also, that the object which makes the call to \prg{deposit} described in (1), and the object which 
has access to \prg{a} in the current state described in (2) need not be the same: It may well be that the
latter passes  a reference to \prg{a} to the former (indirectly), which then makes the call
to \prg{deposit}.

It remains to think about how access to an \prg{Account} may be obtained. This is the remit of assertion (3): 
It says that if at some time in the future of the state restricted to \SF, 
some object \prg{o} which is external has access to some account \prg{a}, and if \prg{a} exists in the 
current state, then in the current state some object 
from \SF~has access to \prg{a}. Where \prg{o} and $\prg{o}'$ may, but need not, be the same object. And where
 $\prg{o}'$ has to exist and have access to \prg{a} in the \emph{current} state, but 
\prg{o} need not exist in the current state -- it may be allocated later.

\vspace{.1cm}

A  holistic  specification for the bank account, then,
would be our original sufficient functional specification from
fig.~\ref{fig:BankSpec} plus the necessary
specifications (1)-(3) from above. %in fig.~\ref{fig:necBank}.   
This holistic specification
permits an implementation of the bank that also provides  \prg{count}
and \prg{notify} methods, even though the specification does not mention either method.
Critically, though, the \Chainmail specification
does not permit an
%specification from fig.~\ref{fig:count},
implementation that includes a \prg{steal} method.
First, the \prg{steal} method clearly changes the balance of
every account in the bank, but assertion (1) requires that any method
that changes the balance of any account must be called \prg{deposit}.
Second, the \prg{steal} method changes the balance of every account in
the system, and will do so without the caller having a reference to
most of those accounts, thus breaching assertion (2).   
% SD chopped. unnecessary
%Note that \prg{steal} putting all the funds into the thief's account
%does not breach policy (2) with respect to the thief's own account,
%because that account is passed in as a parameter to the \prg{steal}
%method, and so the called of the \prg{steal} must have access to that
%account.
 
Assertion (3) gives essential protection when dealing with foreign, untrusted code.
% \sophia{I like the story of this paragraph. Please make more eloquent.}
%\sophia{Assertion (3) is a corollary of (2), and thus actually superfluous -- do we say that? Do we say the story just using (2)?}
%\susan{I think having 1 and 2 would make it more complicated. You could have a footnote though that says actually you don't even need 3 as it is a corollary of 2}
%\james{they're examples. it's not a big thing}
When an \prg{Account} is given out to untrusted third parties, assertion (3) guarantees that
this \prg{Account} cannot be used to obtain access to further  \prg{Account}s. 
The \prg{ACM} does not trust its authors, and certainly does not want
to give them access to \prg{acm\_acc}, which contains all of
the \prg{ACM}'s money. Instead, in order to receive money, it will
pass a secondary account used for incoming funds, \prg{acm\_incoming},
into which an \prg{author}  will pay the fee, and from which the \prg{ACM} will transfer the money
back into the main \prg{acm\_acc}. Assertion (3) is crucial, because it guarantees that even 
malicious authors  could not use knowledge of  \prg{acm\_incoming} to obtain access to
the main \prg{acm\_acc} or any other account. %, to which she did not have access.

\forget{
\sd{Thus, we need to be able to argue, that  passing the \prg{acm\_incoming to some unknown object \prg{u},
with an unknown function \prg{mystery}, is guaranteed not to affect \prg{acm\_acc} , unless \prg{u} already has
access to \prg{acm\_acc} before the call. With holistic assertions we can make this argument formally, while
with traditional specifications we cannot.}}
}
 
\vspace{.1cm} 

In summary, our  necessary specifications are assertions that describe the
behaviour of a module as observed by its clients. 
\forget{in fact, in section XXX\sophia{TO ADD} we show two very different implementations
of the \prg{Bank}/\prg{Account} specification which also satisfy (1)-(3).}
These assertions
 can talk about space, time and control, and thus go beyond
%can be defined and interpreted independently of any particular
%implementation of a specification --- rather our policies constrain
%implementations, in just the same way as traditional functional
%specifications. 
 %This is in contrast to e.g.\ 
 class invariants, which can only
talk   about relations between the  values of the fields of the objects.
%\sophia{This is james' original point, made a bit more concrete. Please check whether it makes sense.}
% of a module.
% invariants across the implementation of an abstract, or
% abstraction functions, which link an abstract model to a concrete
% implementation of that model. 
As our invariants are (usually) independent of concrete implementation details, they do not constrain the code to a
specific implementation.
