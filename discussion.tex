
\paragraph{Design choices}

For our underlying language, we have chosen a class based language; we
use classes, because we concentrate on class-based, object-oriented
programming: we could extend our work to prototype-based programming
by creating an (anonynous) class to reify each
prototype \cite{graceClasses}. We expect the ideas will also be
applicable to other kinds of languages (object-oriented or
otherwise).

We have chosen to use a dynamically langauge because many of the
problems we hope to address are written in dynamically typed
languages: web apps and mashups in Javascript; backends in Ruby or
PHP.  We expect that supporting types would make the problem easier,
not harder, but at the cost of significantly increasing the complexity
of the trusted computing base that we assume will run our programs. In
an open world, without some level of assurance (e.g. proof-carrying
code) about the trustworthiness of type information, unfounded
assumptions about types can give rise to new vulnerabilities that
attackers can exploit \cite{pickles}.

Finally, we don't address inheritance. As a specification language,
individual \Chainmail assertions can be combined or reused without any
inheritance mechanism: the semantics are simply that all
the \Chainmail\ assertions are expected to hold at all the points of
execution that they constrain.  \LangOO\ does not contain inheritance
simply because it is not necessary to demonstrate specifications of
robustness: whether an \LangOO class is defined in one place, or
whether it is split into many multiply-inherited superclasses, traits,
default methods in interfaces or protocols, etc.\ is irrelevant,
provided we can model the resulting (flattened) behaviour of such a
composition as a single logical \LangOO\ class.

\paragraph{Contracts and Preconditions}

Traditional specification lanaguges based on pre- and post-conditions
are generally based on design-by-contract assumptions: ``if the
precondition is not satisfied, the routine is not bound to do
anything'' \cite{meyer92dbc} --- that is, the routine can do
anything. Ensuring preconditions is the responsibility of the caller,
and if the caller fails to meet that responsibility it is not the
responsibility of the invoked function to fix the problem \cite{Mey88}.
Underlying this approach, however, is the assumption of a close
system: that all the modules in a system are equally trusted, and so
inserting redundant tests would just make the system more complex and
more buggy. Indeed, \citet{meyer92dbc} states that ``This principle is
the exact opposite of the idea of defensive programming.''

In an open world, however, we do not have the luxury of trusting the
other components with which we interact --- indeed we barely have the
luxury of trusting ourselves.  We must assume that our methods may be
invoked at any time, with any combination of arguments the underlying
platform will support, irrespective of the overall state of the
system, or of the object that receives the method request: in some
sense, this is the very definition of an open system in an open world.
Since methods cannot control when they are invoked, we must work
as if all (potentially) externally-visible methods just have the precondition
\prg{true} --- and we had better be very careful about any assumptions
we make about which method or objects are in fact externally visible
and which are not.  \kjx{likes this sentence: Holistic specifications
directly support robust programming by making those kinds of
assumptions explicit, giving the necessary conditions under which
objects should be accessesd or their methods invoked, and then can
help programmers ensure their programs maintain those conditions.}

%\footnote{Shall we add this, or is it too technical? 
%It is hypothetical, because it holds only under the hypothesis that all functions have been
%specified with \prg{TRUE} as their preconditions. Namely,  a function is at liberty to behave in any way it
%likes when its precondition does not hold; therefore a function with a precondition different than \prg{TRUE}
%could have a behaviour out
%mean: a function is obligated to deliver its postcondtion if its precondtion holds, but is under no obligation whatsoever when the
%precondtion does not hold. Thus, its beahviour would fall outside the ovals.  }


%%% it's a point James would like made somewhre? but where?



\paragraph{Necessary v.s.\ sufficient conditions.}

%%One might ask whether the necessary conditions are different from
%%the complement of all the sufficient conditions.
%%
%%one might ask indeed. but then one would sound like jacob rees-mogg
Are the necessary conditions the same as the complement of all the
sufficient conditions?  The possible behaviour of a component is the
union of all possible behaviours of each individual function on that
component.  This is shown in the left hand side of
Figure \ref{fig:NecessaryAndSuff}. The all theoretically possible
behaviours are the points in the rectangle.  Each function is a
coloured oval and its possible behaviours are the points in the area
of that oval.  Then, presumably, the necessary conditions are all the
points outside the ovals.

This view is mathematically sound but it is impractical, brittle with
regards to software maintenance, and weak with regards to reasoning in
the open world.


Treating necessary conditions as the complement of the sufficient
conditions is impractical, because it suggests that when interested in
a necessity guarantee a programmer would need to read the
specifications of all the functions in that module.  In view of the
number of these functions, and also the number of behaviours emerging
from their combination, this can be a very large undertaking.  What if
the bank did indeed enforce that only the account owner may withdraw
funds, but had another function which allowed the manager to appoint
an account supervisor, and another which allowed the account
supervisor to assign owners?

This approach is also brittle when it comes to software maintenance,
because it gives no guidance to the team maintaining a piece of
software.  If the necessary conditions are only implicit in the
sufficient conditions, then developers' intentions about those
conditiions are not represented anywhere: there can be no distinction
bewteen a condition that is accidental (if logging is not
implemnented, then it cannot be permitted) and once that is essential
(money can only be transferred by account owners).
Subsequent developers may inadvertenty add functions which break these
intentions, without knowning they've done so.

Finally, this implicit approach is weak when it comes to reasoning
about programs in an open world, because if does not give any
guarantees about objects when they are passed as arguments to calls
into unknown code. For example, what guarantees can we make about the
top of the DOM tree when we pass a wrapper pointing to lower parts of
the tree to an unknown advertisement?
