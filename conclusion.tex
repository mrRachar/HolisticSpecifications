In this paper we have motivated the need for holistic specifications,
presented the \Chainmail specification language for writing such
specifications, and shown how \Chainmail can be used to give holistic
specifications of key exemplar problems: the bank account and the
wrapped DOM.

To focus on the key attributes of a holistic specification language,
we have tried to keep the \Chainmail as presented here as simple as
possible. This has meant our language is intentionally restricted: we
do not even support recursive procedures to avoid circularities in the
metatheory, let alone concurrency, exceptions, distribution,
networking, etc.  We plan to remove these restrictions by applying
techniques such as step-indexing \cite{dd} --- even though this will
necessarily complicate the formalism.
%
We also plan to extend \Chainmail to support reasoning about
conditional trust in programs, and to quantify the risks involved in
interacting with untrustworthy software \cite{swapsies}.

Finally, we hope to develop dynamic monitoring and
automated reasoning techniques to make these kinds of specifications
practically useful.
