{
 ERC20~\cite{ERC20} is a widely used contract on the Ethereum blockchain. It keeps track of participants' tokens;  tokens may be transferred between participants, provided
 the transfer was instigated by the account holder, or somebody authorized by them.
% Traditional specification languages express this requirement implicitly only, while
% we seek to make it explicit. 

Below we give a holistic specification of 
the necessary conditions for a % below says that %~policy \prg{Pol\_ERC20\_tranfer} from below says:  
 decrease in  a participant's balance:  %with an \prg{ERC20}   
 Namely, the decrease (i.e.,  $Next(\prg{e}.\prg{balance}(\prg{p}))$=$...$)  
is caused either by a transfer instigated by the account holder themselves (i.e., $Call(\prg{p},...)$), or by
a transfer instigated by another participant $\prg{p}''$  (i.e., $Call(\prg{p}''...$) who   had   been given authority earlier.
 % by the account holder 
($Authorized(\prg{p},\prg{p}'',\prg{m})$). 
% We show this in Figure 2.

 % \FigAuthorize  

\noindent  
%\prg{Pol\_ERC20\_withdraw} \ $\equiv$ \\ 
\strut \hspace{0.3cm} $\forall \prg{e}:\prg{ERC20}.\forall \prg{p}:\prg{Any}.\forall \prg{m}:\prg{Nat}.$\\
\strut \hspace{0.3cm} $[\ \ \Next{\prg{e}.\prg{balance(p)}}=\prg{e}.\prg{balance(p)}-\prg{m}$ \\ %.\forall\prg{m}:\prg{Nat}.$\\
\strut \hspace{0.4cm} \ \ \ $\longrightarrow$\\
\strut \hspace{0.4cm} $\exists \prg{p}',\prg{p}'':\prg{Any}.$ \\
\strut \hspace{0.4cm} $[\ \  \Calls{\prg{p},\prg{e.transfer(p'.m),\_)}}\  \ \  \vee\, $\\
\strut \hspace{0.4cm} $\ \ \ \ \Calls{\prg{p}'',\prg{e.transferFrom(p,cl',m),\_}}\  \wedge\   \prg{e}.\prg{authorized}(\prg{p},\prg{p}'')=\prg{m}' \ \wedge \prg{m}'\geq \prg{m} \ ] \ \  ]$\\
%\strut \hspace{0.4cm} $\ \ \ \ Authorized(\prg{e},\prg{p},\prg{p}'',\prg{m}) \ ] \ \  ]$\\
\strut \hspace{0.3cm} $] $
%\strut \hspace{0.5cm} \ \ \ $\prg{d}.\prg{ether}\geq \prg{m}\ \wedge$ $\ Fut(Call(\prg{d.send(p)},m))\ \ ] $ 

\noindent
In summary, any decrease of balance is instigated either by the account holder or by somebody authorized to make such a transfer on their behalf. Note the term \prg{e}.\prg{authorized}(\prg{p},\prg{p}''); this means that the 
ERC20 variable \prg{e} holds a field called \prg{auhtorized} of   mapping type, which maps pairs of participants to numbers; such
mappings are supported in Solidity\cite{Solidity,}. % allow for fields which are mappings, as supported 
We use the undescrore ($\_$) to indicate some value of no importance, \eg $\Calls(\prg{p}'',\prg{e.transferFrom(p,cl',m),\_})$ 

We now define what it means for $\prg{p}'$ to be authorized to spend \prg{m} tokens on behalf of $\prg{p}$: At some point in the
past,  \prg{p} gave authority to $\prg{p}'$  to spend on their behalf  a number %of tokens  
which is larger or equal   \prg{m} 
plus the sum of  tokens 
spent so far by $\prg{p}' $ on the behalf of \prg{p}. We use \hlc[yellow]{yellow highlights for informal explanations.}

\vspace{.06cm}
\noindent
$
\begin{array}{lcl}
 \prg{e}.\prg{authorized}(\prg{p},\prg{p}'',\prg{m})
&
 \triangleq 
&
  \left\{
                            \begin{array}{l}
                              \Prev(\Calls(\prg{p},\prg{e.authorize(p,m)\ )}\     \\
                             \strut \ \  \mbox{\hlc[yellow]{In the previous step, {\tt p} authorized {\tt p'} to spend up to  {\tt m}}}\\
                               \vee \\
                           \Prev(\ \prg{e}.\prg{authorized}(\prg{p},\prg{p}'',\prg{m})\ ) \,  \,   \wedge\,   \\
                               \ \  \neg (\Prev(\Calls(\prg{p}',\prg{e.transferFrom(p,p',\_)},\_ ) ) )\\
                               \strut \ \  \mbox{\hlc[yellow]{In the previous step, {\tt p'} was authorized for  {\tt m}}}\\
                               \strut \ \  \mbox{\hlc[yellow]{and did not transfer on behalf of  {\tt p}}}\\
                               \vee \\
                              \Prev(\, \prg{e}.\prg{authorized}(\prg{p},\prg{p}'',\prg{m})\,  \,   \wedge\, \prg{m}\leq \prg{m}'-\prg{m}''\, \wedge \, \\
                               \ \  \Prev(\Calls(\prg{p}',\prg{e.transferFrom(p,cl'',m'')},\_ ) ) 
                               \\
                                  \strut \ \  \mbox{\hlc[yellow]{In the previous step, {\tt p'} was authorized for  {\tt m'}}}\\
                               \strut \ \  \mbox{\hlc[yellow]{and transfered an amount smaller than   {\tt m'-m}}}
                        \end{array} 
                         \right.
 \end{array}
 $                        
 
 \vspace{.09cm}  
 
 \noindent
 
  
 %on \prg{p}'s behalf 
%
%Wrt transfer of tokens, the  \RoSpec~specification from 





%\vspace{.05cm}

 This holistic specification  gives  to  account holders an "authorization-guarantee": their balance cannot decrease unless they themselves,  or somebody they had  authorized, instigates a transfer of tokens. Moreover, authorization is not transitive.
 
% \FigEscrow 
%\vspace{-.2in}
\paragraph{Comparison with Traditional Specifications} 
 %
 Traditional  specifications %for the ERC20 example would consist of 
 describe the behaviour of each function separately even for a single object. 
 They  give pre- and postconditions, % for a function,
  and thus describe a  {\em sufficient} condition for the effect of the particular function. 
% \noindent
%
%As stated earlier,  traditional specifications give {\em sufficient} conditions, 
 e.g., if \prg{p''}  is authorized and executes \prg{transferFrom}, then   the balance decreases. But they are {\em implicit} about the overall behaviour and the   {\em necessary} conditions, 
e.g., what are all the possible actions that can cause a decrease of balance?
%effect (e.g, a decrease of balance) can only take place if the necessary conditions hold (e.g., the owner called \prg{transfer} or somebody authorized  called  \prg{transferFrom}). 

  
With traditional  specifications, to obtain the "authorization-guarantee", one would need to inspect the pre- and post- conditions of {\em all} the functions 
in the contract, and determine which ones decrease balances, and then determine which ones affect authorizations. 
Moreover, with traditional  specifications, nothing stops the next release of the contract to add, e.g., a method which 
allows participants to share their authority, and thus
 violate the "authorization-guarantee".
 }
