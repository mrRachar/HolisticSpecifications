 
 ERC20~\cite{ERC20} is a widely used token standard which describes the 
 basic functionality expected by any    Ethereum-based token contract. 
 It issues and keeps track of participants' tokens, and supports the  transfer
 of tokens between participants. 
%
%
%An important question, therefore, is to identify the precise circumstances under which a transfer of tokens may take place.
%The answer is that 
Transfer of tokens 
 can   take place only provided that  there were sufficient tokens in the
 owner's account, and that
 the transfer was instigated by the owner, or by somebody authorized
 by the owner.

We specify this in \Chainmail as follows:
%
%can use holistic specifications to express our understanding of
%these circumstances, by specifying the necessary conditions for a
%decrease in a participant's balance, and the necessary conditions for
%authorization.
%
A decrease in  a participant's \prg{balance} 
%(\ie  $\prg{e}.\prg{balance}=...\, \wedge\, \Next{\prg{e}.\prg{balance}=...}$)
can only be caused by a transfer instigated by the 
account holder themselves (\ie $\Calls {\prg{p}} {\prg{transfer}} {...} {...}$), or by
an authorized transfer instigated by another participant $\prg{p}''$  (\ie $\Calls {\prg{p}''} {{\prg{transferFrom}} } {..} {..}$) who 
has authority for more than the tokens spent(\ie  $\prg{e}.\prg{allowed}(\prg{p},\prg{p}'')\geq \prg{m}'$)
%This is described by the following policy, which
% binds any \prg{e} which is an \prg{ERC20}  contract:

\vspace{.15cm}
\noindent
% \strut \hspace{0.3cm} 
$\forall \prg{e}:\prg{ERC20}.\forall \prg{p}:\prg{Object}.\forall \prg{m},\prg{m}':\prg{Nat}.$\\
\strut \hspace{0.3cm} $[\ \ \prg{e}.\prg{balance}=\prg{m}+\prg{m'}\ \wedge \ \Next{\prg{e}.\prg{balance}=\prg{m}'}$ \\ %.\forall\prg{m}:\prg{Nat}.$\\
\strut \hspace{0.4cm} \ \ \ $\longrightarrow$\\
\strut \hspace{0.4cm} \ \ \ $\exists \prg{p}',\prg{p}'':\prg{Object}.$ \\
\strut \hspace{0.4cm} \ \ \  $[\ \  \Calls{\prg{p}} {\prg{transfer}}  {\prg{e}}  {\prg{p}',\prg{m}} \  \  \ \vee\, $\\
\strut \hspace{0.4cm} \ \ \   $\ \ \ \ \prg{e}.\prg{allowed}(\prg{p},\prg{p}'')\geq \prg{m} \ \wedge \ \Calls{\prg{p}''} {\prg{transferFrom}}  {\prg{e}}  {\prg{p}',\prg{m}}\       \  ]$\\
\strut \hspace{0.3cm} $] $
\vspace{.15cm}

\noindent
That is to say: if next configuration witnesses a decrease of the balance by
 $\prg{m}'$, then the current configuration was a call of \prg{transfer} instigated by
 the owner, or  call of \prg{transferFrom} instigated by somebody authorized.
The term \prg{e}.\prg{allowed}(\prg{p},\prg{p}''), which means that the
ERC20 variable \prg{e} holds a field called \prg{allowed}   which maps pairs of participants to numbers; such
mappings are supported in Solidity\cite{Solidity}.
 
We now define what it means for $\prg{p}'$ to be authorized  to  spend 
up to \prg{m} tokens on  $\prg{p}$'s behalf: At some point in the
past,  \prg{p} gave authority to $\prg{p}'$  to spend   \prg{m}
plus the sum of  tokens
spent so far by $\prg{p}' $ on the behalf of \prg{p}. 

 
\vspace{.15cm}
\noindent
 $\forall \prg{e}:\prg{ERC20}.\forall \prg{p},\prg{p'}:\prg{Object}.\forall \prg{m}:\prg{Nat}.$\\
\strut \hspace{0.3cm} $[\ \ \prg{e}.\prg{allowed}(\prg{p},\prg{p}')=\prg{m} $\\
\strut \hspace{0.4cm} \ \ \ $\longrightarrow$\\
\strut \hspace{0.4cm} \ \ \  
     $\PrevId\langle\ \  \Calls{\prg{p}}  {\prg{approve}}  {\prg{e}} {\prg{p}',\prg{m}} $\\
      \strut \hspace{1.7cm} \ $\vee $\\
\strut \hspace{1.7cm} \  
     $    \prg{e}.\prg{allowed}(\prg{p},\prg{p}')=\prg{m}   
        \  \wedge\ $\\
\strut \hspace{1.5cm} \ \ \ \ \          
$  \neg   (\, {\Calls{\prg{p}'} {\prg{transferFrom}} {\prg{e}} {\_} {\_} } \, \vee \, {\Calls{\prg{p}} {\prg{approve}} {\prg{e}} {\prg{p},\_} } \, ) $\\
      \strut \hspace{1.7cm}\  $\vee $\\
\strut \hspace{1.7cm}   \  $ \exists \prg{p}'':\prg{Object}.\exists\prg{m'}:\prg{Nat}.$\\
 \strut \hspace{1.7cm}\  $[\   
  \prg{e}.\prg{allowed}(\prg{p},\prg{p}')=\prg{m}+\prg{m}'  \, \wedge\,   {\Calls{\prg{p}'} {\prg{transferFrom}} {\prg{e}} {\prg{p}'',\prg{m}'}  }   ]$\\
\strut \hspace{0.4cm} \ \ \  \ \ \  \ \ \ \ \ $\rangle $\\
\strut \hspace{0.3cm} $]$
\vspace{.15cm}
 
%In more detail, $\prg{p}'$ is allowed to spend 
%up to \prg{m} tokens on their behalf of $\prg{p}$, if in the immediately previous step either a)
% \prg{p} made the call \prg{approve} on \prg{e} 
%with arguments $\prg{p}'$ and \prg{m}, or b)  
%$\prg{p}'$ was allowed to spend  up to \prg{m} tokens for $\prg{p}$
%and did not transfer any of \prg{p}'s tokens, nor did \prg{p} issue a fresh authorization,
%or c) \prg{p} was authorized for $\prg{m}+\prg{m}'$ and spent $\prg{m}'$ 
%  
 
 Thus, the holistic specification gives to account holders an
 "authorization-guarantee": their balance cannot decrease unless they
 themselves, or somebody they had authorized, instigates a transfer of
 tokens. Moreover, authorization is {\em not} transitive: only the
 account holder can authorise some other party to transfer funds from
 their account: authorisation to spend from an account does not confer
 the ability to authorise yet more others to spend also.
 
% \paragraph{Comparison with Traditional Specifications}
 
 With traditional  specifications, to obtain the "authorization-guarantee", 
one would need to inspect the pre- and post- conditions of {\em all} the functions
in the contract, and determine which of the functions decrease balances, and which of the functions 
 affect authorizations.
 In the case of the \prg{ERC20}, one would have to inspect all eight such specifications
 (given in appendix \ref{ERC20:appendix}), 
 where only five are relevant to the question at hand.
 In other cases, \eg the DAO, the number of the functions which are unrelated
 to the question at hand can be very large.
  
More importantly, with traditional  specifications, nothing stops the next release of the contract to add, 
\eg, a method which allows participants to share their authority, and thus
violate the "authorization-guarantee", or even a super-user from skimming 0.1\% from each of the accounts.

