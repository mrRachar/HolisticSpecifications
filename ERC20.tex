 
 ERC20~\cite{ERC20} is a widely used token standard which describes the 
 basic functionality expected by any    Ethereum-based token contract. 
 It issues and keeps track of participants' tokens, and supports the  transfer
 of tokens between participants. 


An important question, therefore, the circumstances under which a transfer of tokens may take place.
The answer is that transfer of tokens 
 can only take place only provided that  there were sufficient tokens in the
 owner's account, and that
 the transfer was instigated by the owner, or by somebody authorized by them.
 
We can answer this question with the holistic specification from below, which mandates 
the necessary conditions for necessary conditions for a 
 decrease in  a participant's balance, and the necessary conditions for authorization.

Namely, a decrease in  a participant's balance 
(\ie  $\prg{e}.\prg{balance}=...\, \wedge\, \Next{\prg{e}.\prg{balance}=...}$)
is caused either by a transfer instigated by the 
account holder themselves (\ie $\Calls{\prg{p}, ...}$), or by
a transfer instigated by another participant $\prg{p}''$  (\ie $\Calls{\prg{p}''...}$) who 
has authority for more than the tokens spent(\ie  $\prg{e}.\prg{allowed}(\prg{p},\prg{p}'')\geq \prg{m}'$)
This is described by the following invariant, which
is expected from any \prg{e} which is an \prg{ERC20}  contract:

\vspace{.15cm}
\noindent
% \strut \hspace{0.3cm} 
$\forall \prg{e}:\prg{ERC20}.\forall \prg{p}:\prg{Object}.\forall \prg{m},\prg{m}':\prg{Nat}.$\\
\strut \hspace{0.3cm} $[\ \ \prg{e}.\prg{balance}=\prg{m}+\prg{m'}\ \wedge \ \Next{\prg{e}.\prg{balance}=\prg{m}'}$ \\ %.\forall\prg{m}:\prg{Nat}.$\\
\strut \hspace{0.4cm} \ \ \ $\longrightarrow$\\
\strut \hspace{0.4cm} \ \ \ $\exists \prg{p}',\prg{p}'':\prg{Object}.$ \\
\strut \hspace{0.4cm} \ \ \  $[\ \  \Calls{\prg{p},\prg{e.transfer(p',m)},\_} \  \  \ \vee\, $\\
\strut \hspace{0.4cm} \ \ \   $\ \ \ \ \prg{e}.\prg{allowed}(\prg{p},\prg{p}'')\geq \prg{m} \ \wedge \ \Calls{\prg{p}'',\prg{e.transferFrom(p,p',m),\_}}\       \  ]$\\
\strut \hspace{0.3cm} \ \ \ $] $
\vspace{.15cm}

\noindent
In more detail, if the next configuration witnesses a decrease of the balance by
 $\prg{m}'$, then the current configuration was a call of \prg{transfer} instigated by
 the owner, or it was a call of \prg{transferFrom} instigated by somebody authorized.
Note the term \prg{e}.\prg{allowed}(\prg{p},\prg{p}''), which means that the
ERC20 variable \prg{e} holds a field called \prg{allowed} of   mapping type, which maps pairs of participants to numbers; such
mappings are supported in Solidity\cite{Solidity}, but could also be understood as ghost fields or predicates\footnote{cite ghost field}.
Also, % allow for fields which are mappings, as supported
we use an underscore ($\_$) to indicate some value or variable of no importance; thus the term $\Calls{\prg{p}'',\prg{e.transferFrom(p,cl',m),\_}}$ is a convenient shorthand for 
$\exists \prg{m}''.\Calls{\prg{p}'',\prg{e.transferFrom(p,cl',m),m''}}$.

We now define what it means for $\prg{p}$ to have authorized $\prg{p}'$ to  spend 
up to \prg{m} tokens on the behalf of $\prg{p}$: At some point in the
past,  \prg{p} gave authority to $\prg{p}'$  to spend   \prg{m}
plus the sum of  tokens
spent so far by $\prg{p}' $ on the behalf of \prg{p}. 

 
\vspace{.15cm}
\noindent
 $\forall \prg{e}:\prg{ERC20}.\forall \prg{p},\prg{p'}:\prg{Object}.\forall \prg{m}:\prg{Nat}.$\\
\strut \hspace{0.3cm} $[\ \ \prg{e}.\prg{allowed}(\prg{p},\prg{p}')=\prg{m} $\\
\strut \hspace{0.4cm} \ \ \ $\longrightarrow$\\
\strut \hspace{0.4cm} \ \ \  
     $\PrevId\langle\ \  \Calls{\prg{p},\prg{e}.\prg{approve}(\prg{p}',\prg{m})} $\\
      \strut \hspace{1.7cm} \ $\vee $\\
\strut \hspace{1.7cm} \  
     $    \prg{e}.\prg{allowed}(\prg{p},\prg{p}')=\prg{m}   
        \  \wedge\ $\\
\strut \hspace{1.5cm} \ \ \ \ \          $  \neg   (\, \Calls{\prg{p}',\prg{e.transferFrom(p,\_,\_)},\_ }\, \vee \, \Calls{\prg{p},\prg{e}.\prg{approve}(\prg{p}',\_)\ }\, ) $\\
      \strut \hspace{1.7cm}\  $\vee $\\
\strut \hspace{1.7cm}   \  $ \exists \prg{p}'':\prg{Object}.\exists\prg{m'}:\prg{Nat}.$\\
 \strut \hspace{1.7cm}\  $[\   
  \prg{e}.\prg{allowed}(\prg{p},\prg{p}')=\prg{m}+\prg{m}'  \, \wedge\,   \Calls{\prg{p}',\prg{e.transferFrom(p,p'',\prg{m}')},\_ }\  ]$\\
\strut \hspace{0.4cm} \ \ \  \ \ \  \ \ \ \ \ $\rangle $\\
\strut \hspace{0.3cm} $]$
\vspace{.15cm}
 
In more detail, $\prg{p}'$ is allowed to spend 
up to \prg{m} tokens on their behalf of $\prg{p}$, if in the immediately previous step either a)
 \prg{p} made the call \prg{approve} on \prg{e} 
with arguments $\prg{p}'$ and \prg{m}, or b)  
$\prg{p}'$ was allowed to spend  up to \prg{m} tokens for $\prg{p}$
and did not transfer any of \prg{p}'s tokens, nor did \prg{p} issue a fresh authorization,
or c) \prg{p} was authorized for $\prg{m}+\prg{m}'$ and spent $\prg{m}'$ 
  
 

 Thus, the holistic specification  gives  to  account holders an "authorization-guarantee": their balance cannot decrease unless they themselves,  or somebody they had  authorized, instigates a transfer of tokens. Moreover, authorization is {\em not} transitive.
 
\paragraph{Comparison with Traditional Specifications}
 
 Traditional  specifications %for the ERC20 example would consist of
 describe the behaviour of each function separately.
 They  consist of pre- and post- conditions for each function; the pre-condition is 
 a {\em sufficient} condition for the effect of the particular function.
 Usually, for each function, we need one specification for the case 
 where the function's pre-condition holds, another for the case where it does not.\footnote{ARGH! 
 Here we use pre-condition with two different meanings! What to do?}
  
With traditional  specifications, to obtain the "authorization-guarantee", 
one would need to inspect the pre- and post- conditions of {\em all} the functions
in the contract, and determine which of the functions decrease balances, and which of the functions 
 affect authorizations.
 In the case of the \prg{ERC20}, one would have to inspect all eight such specifications, 
 where only five are relevant to the question at hand.
 In other cases, \eg the DAO, the number of the functions which are unrelated
 to the question at hand can be very large.
 Moreover, with the traditional specification, the "authorization-guarantee" can only be 
 obtained if we all the functions have a pre- and post-conditions, while this is not necessaty for the
 holistic case.
 
More importantly, with traditional  specifications, nothing stops the next release of the contract to add, 
\eg, a method which allows participants to share their authority, and thus
violate the "authorization-guarantee", or even a super-user from skimming 0.1\% from each of the accounts.

In Figure \ref{fig:classicalERC20} we outline a traditional specification for the \prg{ERC20}.
We give two speficiations for \prg{transfer}, another two for \prg{tranferFrom}, and one for all 
the remaining functions. The  first specification says, \eg, that if  
 \prg{p} has sufficient tokens, and it calls \prg{transfer}, then the transfer will take place.  
The second specification says that  if \prg{p} has insufficient tokens, then 
the transfer will not take place (we assume that in this
specification language, any entities not mentioned in the pre- or post-condition 
are not affected).\footnote{cite framing -- but anyway, our subject is not the classical spec.}
 
 Similarly, we would have to give another two specifications to define the behaviour of 

if \prg{p''} is authorized and executes \prg{transferFrom}, then   the balance decreases. 
But they are {\em implicit} about the overall behaviour and the   {\em necessary} conditions,
e.g., what are all the possible actions that can cause a decrease of balance?


 
\begin{figure}   
\fbox{
$
\begin{array}{c}
 \prg{e}:\prg{ERC20}\ \wedge\  \prg{p},\prg{p''}:\prg{Object} 
  \wedge\ \prg{m},\prg{m}',\prg{m}'':\prg{Nat}\ \wedge\   \\
 \prg{e}.\prg{balance(p)} = \prg{m}+\prg{m}'\ \ \wedge\ \ \prg{e}.\prg{balance(p'')} = \prg{m}''\ \ \wedge\ \ \prg{this}=\prg{p} \\
   \{ \ \ \prg{e.transfer(p'',m')} \ \ \}   \\
    \prg{e}.\prg{balance(p)} = \prg{m}\ \ \wedge\ \ \prg{e}.\prg{balance(p'')} = \prg{m}''+\prg{m}'
\ \ \\
\ \ \\
  \prg{e}:\prg{ERC20}\ \wedge\  \prg{p},\prg{p'}:\prg{Object}  \wedge\ \prg{m},\prg{m}',\prg{m}'':\prg{Nat}\ \wedge\     \prg{e}.\prg{balance(p)} = \prg{m} \ \ \wedge \prg{m} <  \prg{m}'  \\
   \{ \ \ \prg{e.transfer(p',m')} \ \ \}   \\
  \prg{e}.\prg{balance(p)} = \prg{m}  
  \\
  \\
\prg{e}:\prg{ERC20}\ \wedge\  \prg{p},\prg{p'},\prg{p}'':\prg{Object} 
  \wedge\ \prg{m},\prg{m}',\prg{m}'',\prg{m}''':\prg{Nat}\ \wedge\   \\
 \prg{e}.\prg{balance(p)} = \prg{m}+\prg{m}'\ \ \wedge\ \ \prg{e}.\prg{allowed(p,p')}=\prg{m}'''+\prg{m}' \ \wedge\\
  \prg{e}.\prg{balance(p'')} = \prg{m}''\ \ \wedge\ \ \prg{this}=\prg{p'} \\
   \{ \ \ \prg{e.transferFrom(p',p'',m')} \ \ \}   \\
    \prg{e}.\prg{balance(p)} = \prg{m}\ \ \wedge\ \ \prg{e}.\prg{balance(p'')} = \prg{m}''+\prg{m}'
     \ \wedge\ \ \prg{e}.\prg{allowed(p,p')}=\prg{m}'''
\ \ \\
\ \ \\
  \prg{e}:\prg{ERC20}\ \wedge\  \prg{p},\prg{p'}:\prg{Object}  \wedge\ \prg{m},\prg{m}',\prg{m}'':\prg{Nat}\ \wedge\ \prg{this}=\prg{p}' \ \wedge \\
      ( \ \prg{e}.\prg{balance(p)} =\prg{m} \wedge \prg{m} <  \prg{m}''\  \vee \ 
  \prg{e}.\prg{allowed(p,p')}=\prg{m'} \wedge \prg{m'} < \prg{m}'' \ ) \\
   \{ \ \ \prg{e.transferFrom(p,p'',m'')} \ \ \}   \\
  \prg{e}.\prg{balance(p)} = \prg{m} \wedge  \prg{e}.\prg{allowed(p,p')}=\prg{m'}
  \\
  \\
  \prg{e}:\prg{ERC20}\ \wedge\  \prg{p},\prg{p'}:\prg{Object}  \wedge\ \prg{m}:\prg{Nat}\ \wedge\ \prg{this}=\prg{p}  \\
   \{ \ \ \prg{e.approve(p',m')} \ \ \}   \\
  \prg{e}.\prg{allowed(p,p')} = \prg{m} 
  \\
  \\
   \prg{e}:\prg{ERC20}\ \wedge\ \prg{m}:\prg{Nat}\ \wedge\    \prg{p}.\prg{balance}=\prg{m}    \\
   \{ \ \ \prg{k}=\prg{e.balanceOf(p)} \ \ \}   \\
  \prg{k}=\prg{m} \ \wedge \ \prg{e.balanceOf(p)} = \prg{m}  
  \\
  \\
   \prg{e}:\prg{ERC20}\ \wedge\ \prg{m}:\prg{Nat}\ \wedge\    \prg{e}.\prg{allowed(p,p')}=\prg{m}    \\
   \{ \ \ \prg{k}=\prg{e.allowance(p,p')} \ \ \}   \\
  \prg{k}=\prg{m} \ \wedge \ \prg{e}.\prg{allowed(p,p')}=\prg{m} 
  \\
  \\
   \prg{e}:\prg{ERC20}\ \wedge\ \prg{m}:\prg{Nat}\ \wedge\     \sum_{\prg{p}\in dom(\prg{e}.\prg{balance})}^{}{\prg{e}.\prg{balance}(\prg{p})}=\prg{m}    \\
   \{ \ \ \prg{k}=\prg{e.totalSupply()} \ \ \}   \\
  \prg{k}=\prg{m}   
\end{array}
$
}
\caption{Classical specification for the \prg{ERC20}}
\label{fig:classicalERC20}
\end{figure}
